\chapter{Patient cases}

Case 1
Eva Andersen skoleelev hatt insulinkrevende diabetes mellitus 3 år.
bror diabetes brukt insulin flere år.
bruker insulinpenn hurtigvirkende insulin injiserer huden hvert måltid dessuten injeksjon langtidsvirkende insulin kvelden.
lite motivert håndtere sukkersyke.
uteblir kontroller, delvis tatt insulin periodevis ignorert kostråd
synes leit fått sukkersyke
Eva siste døgn følt tungpusten, kvalm kastet opp.
Rødmusset.
Hurtig pust.
Tørr pannen.
Acetonlukt.
Normalt blodtrykk.
delvis uklar, vurderer henvisning sykehus.

Case 2
Pasienten 56 år gammel mann stort sett frisk tidligere bortsett meniskplager ført meniskoperert knær.
arbeidet avløser jordbruk/skogbruk.
attføring kneplager.
Lungepoliklinikken hatt tungpust siste 3 - 4 månedene.
tungpust hvile, tung pusten bakker trapper , stoppe hvile gått 400 -- 500 meter flat mark, kortere distanser.
tillegg tungpust hatt følelsen tett brystet.
hostet mye, fått del slim grønt farge føler "der noe" brystet.
lagt merke "skrål piping" brystet.
symptomene hatt omtrent lenge følt verre pusten.
nærmere utspørring kommer nok tyngre pusten siste par år.
merket fått problemer holde følge jevnaldringer situasjoner tidligere gått greitt festet spesielt brakt bane.
Imidlertid inntrådt merkbar forverring pusten siste 3 -- 4 måneder.
aldri plaget astma allergier tidligere livet.
røkt ca.
40 år, mesteparten tiden 3 pakker rulletobakk uka, reduserte pakke uka par år sluttet helt 3 måneder pusteplager.
undersøkelsen Lungeavdelingen ubesværet pusten hvile.
lytting stetoskop lungeflatene høres unormale lyder ("pipelyder") puster (i ekspiriet), mindre grad puster inn.
Hjertelydene normale.
Pulsen 80 regelmessig.
Røntgenbildet lungene normalt.
gjort spirometri viser VK (vitalkapasiteten volumet luft maksimalt fylle lungene pust) 2,56 liter (53\% normalt) FEV1 (det forserte ekspiratoriske volum volumet blåse lungene sekund) 1,28 liter (34\% normalt).
gitt behandling antibiotika behandlingen bricanyl Atrovent (som utvider bronkiene ) pulmicort (et kortisonpreparat inhalerer) hvert betydelig bedre idet pusten lettere, hosten bedre grønnfargete oppspyttet forsvant gjenvant vante tilstand.
spirometriundersøkelse viste måned VK 3,20 liter (66\% normalt) FEV1 2,02 liter (53\% normalt), altså betydelig bedring.

Case 3
Hanne 9.
klasse ungdomsskolen, to yngre søsken.
tidligere betydning, aldri hospitalisert.
morgen februar klager Hanne føler syk, litt vondt halsen, hodepine smerter hele kroppen temperatur 38,7 °C.
del influensa distriktet tror foreldrene Hanne holder utvikle.
holder hjemme skolen dårligere utover dagen.
far kommer hjem 16-tiden, Hanne litt omtåket, ligger døs temperaturen måles 40,9 °C.
undersøkelsen finner legen desorientert, nakkestiv spredt kroppen petekkier ("småblødninger" huden).
Blodtrykket måles 105/70 mm Hg.
Hanne innlegges umiddelbart Barneklinikken mistanke "smittsom hjernehinnebetennelse".
mottagelsen sykehuset Hanne undersøkt vakthavende lege, henges intravenøs væsketillførsel, blodprøves takes gjøres spinalpunksjon.
Spinalvæsken tydelig blakket.
2 blodkulturer tatt, startes umiddelbart intravenøs antibiotika behandling Penicillin G Kloramfenikol.
par timer innkomsten får Hanne par minutters krampeanfall, behandles diazepam intravenøst.
besluttes Hanne intuberes legges respirator.
Dagen innkomst rapporterer mikrobiologisk laboratorium funn Gram-negative diplokokker spinalvæske blodkulturer, kommer oppvekst Neisseria meningitidis serogruppe C.
Hannes to yngre søsken får penicillintabletter uke (Helsedirektoratets forskrift).
Kommunelegen setter igang vaksinasjon hele Hannes familie, nære venner skoleklasse.

Case 4
Trond Øvrebotten, 42 år, oppsøker fastlegen 2-3 måneder følt ubehag, trykk brystet anstrengelse.
gang kjent opphisselse.
røyker 10-15 sigaretter daglig, trener spiser sunn mat.
far døde plutselig ca.
50 år gammel.
Legen starter medikamentell behandling pasientens symptomer angina pectoris henviser spesialistundersøkelse.
rekker komme innlegges sykehuset grunn kraftige brystsmerter litt bedre nitroglycerin, helt bort.

Case 5
Pasienten 64 år gammel kvinne tidligere stort sett frisk.
siste 5 månedene merket følelse ufullstendig tømming avføring.
flere anledninger spor friskt blod avføringen.
tilskriver hemorroider hatt før, ventet over.
egen finner galt vanlig undersøkelse.
rektaleksplorasjon (kjenne endetarmsåpningen finger) kjenner legen vidt kanten uregelmessighet tarmveggen.
Legen ser spor gamle ytre hemorroider sannsynlig blødningskilde nå.
Prøver blod avføringen positive.
Blodprøver viser lett blodmangel (hemoglobin 10.
8; normalt kjønn alder >12).
Øvrige blodprøver normale.
Pasienten henvist colonoscopi (endoskopisk undersøkelse tykktarmen).

Case 6
første konsultasjon funnet forstørrede tonsiller hvitlig belegg forstørrede glandler sider halsen.
tatt halsprøve strept.
test positiv.
fremkom kjæresten nylig gjennomgått streptokokktonsilitt.
startet behandling peroral penicillin (Apocillin) vanlig dosering.
Pasienten kommer konsultasjon manglende effekt behandlingen penicillin.
verre, fikk svelgbesvær, fikk fast føde, besværligheter væske.
Samtidig vedvarende slapp dårlig allmenntilstand.

Case 7
Berta fikk påvist cancer mamma våren fem år siden.
påvist metastaser skjelett innlagt ortopedisk avdeling fractura colli femoris behandlet innsetting totalprotese.
hensyn postoperativt forløp opptrening hoften beskrives pasienten minste problem.
hensyn smerter sier uttalt halebenet.
smerter høyere opp, tilsvarende os sacrum.
stemmer godt ferske funn rtg.
bekken.
Smertene føles trykkende karakter.
utstrålende smerter tegn overfølsomhet overliggende hudområde.
spesiell palpasjonsømhet muskulatur.
Således holdepunkter nevrogen muskulært betingede smerter.
Pasienten svært tilbakeholden forbruk analgetika.
Bruker Napren E 250 mg x 2 samt Paracet behov.
enig fortsetter Napren E uendret dose.
angir intoleranse Paralgin Forte, unngår moderat virkende opioider gir heller foreløpig Ketorax 5 mg behov.
Forklarer bør lav terskel ta Ketorax.
justere enkeltdosene 1/2-2 tabletter, føler gir effekt uakseptable bivirkninger.
Legger vekt smertelindring gi vesentlig økt livskvalitet.
tidligere prøvd morfin, svært kvalm dette.
Sannsynligvis tolerere «fast serumspeil» morfin form Dolcontin; tross kvalm intramuskulære intravenøse støtdoser.
orientert forholdsvis kort utvikler toleranse sederende virkning opioider.

Case 8
Thomas 8.
klasse aktiv gutt fotball volleyball.
par dager vondt halsen, dårlig allmenntilstand feber tilkaller foreldrene legevakt.
Foreldrene frykter Thomas halsinfeksjon lurer bør antibiotika (Thomas reise leirskole forbindelse konfirmasjons-forberedelsene 3 dager).
undersøkelse pharynx finner legen tonsillene forstørrete, ses hvitlig belegg tonsillene.
hovne lymfeknuter submandibulært.
Legen besluttet starte penicillin-behandling (tabletter Apocillin 0,5 + 0,5 + 1 mill IE) tatt halsprøve innsendes bakteriologisk dyrking.
Tredje dag legevaktbesøk Thomas fortsatt omkring 39°C kommer undersøkelse legekontoret.
undersøkelse finnes økende tonsilleforstørrelse - møtes nesten midtlinjen.
Tonsillene belagt, ses flere petekkier (småblødninger) bløte gane.
stygg foetor ex ore.
Pasienten puster gjennom munnen tett nesen.
Telefonhenvendelse mikrobiologisk laboratorium avklarer innsendte halsprøve viser moderat hemolytiske streptokokker gr C blandet halsflora.

