\documentclass[a4paper, 11pt]{article}
\usepackage[T1]{fontenc}
\usepackage[utf8]{inputenc}
\usepackage[english]{babel}
\usepackage{textcomp} % Euro symbols etc
\usepackage{graphicx} % support graphics
\usepackage{hyperref} % links in the document
\usepackage{float} % position of figures
\usepackage{paralist} % inline lists
\usepackage{verbatim} % multi-line comments
\usepackage{listings} % Syntax colored code
\usepackage{booktabs} % Professional tables
\usepackage{tabularx} % Simple column stretching
\usepackage{multirow} % Row spanning
\usepackage{wrapfig} % Wrap text around figures
\usepackage{amsmath} % Advanced maths
\usepackage{array}
\usepackage[normalsize, bf]{caption}
\usepackage{color}
\usepackage{textcomp}
\usepackage{fixltx2e}

\setcounter{tocdepth}{2} % Depth of table of contents

% Configure links in pdfs
\hypersetup{
    bookmarksopen=false, % Hide bookmarks menu
    colorlinks=true, % Don't wrap links in colored boxes
%    pdfborder={0 0 0} % Remove ugly boxes
}


%============
% Top matter
%============
\title{TDT4215 Web-intelligence\\Project task 4: evaluation}
\author{Group 1:\\Even Wiik Thomassen, Terje Snarby, Weilin Wang}
\date{\today}

\begin{document}
\maketitle
\tableofcontents


%=====================
\section{Introduction}
%=====================
This paper presents task 4 of the project in TDT4215 Web-intelligence at
Norwegian University of Science and Technology (NTNU). It will
describe the method we used to solve task 3 (\autoref{sec:method}),
the results from task 3 (\autoref{sec:result}),
and evaluation of the results (\autoref{sec:evaluation}).


%===============
\section{Method}
%===============
\label{sec:method}
This section describe methods used to solve task 3.

\subsection{Preprocessing and parsing}
%-------------------------------------
Patient cases were provided as a Word file, which included eight cases. We
created a text file for each case, and made sure they were in utf-8 charset.

Therapy chapters from ``Norsk legemiddelhåndbok'' were provided as HTML files,
invalid html5 files in iso-8859-1 charset. We first preprocessed these files
by removing some of the HTML tags to make them easier to parse, and we
converted them to utf-8 charset.

We created a custom parser for parsing therapy chapters, based on Python's
HTMLParser. We parse one HTML file at a time, creating Therapy objects for
each chapter or sub*-chapter. The text found in these chapters are stored
on the objects. We stored links as a list on each object, while we preserved
their text in the object text. Sections which list relevant drugs were removed
from the text but stored as they might be useful later.

We stored the output of parsing in JSON format, so they could be saved to disk
between runs and loaded quickly.

\subsection{Stopwords}
%---------------------
To reduce the number of terms in documents, and to remove words which provide
little or no relevant information value, we removed stopwords.
We used a list of Norwegian stopwords in both ``bokmål'' and ``nynorsk'' which we
found online\footnote{Stopword source: \url{http://www.wisweb.no/999/147/33899-170.html}},
and we added a few words ourself. A complete list of these stopwords can
be found in \autoref{tab:stopwords} in \autoref{appendix}.

\subsection{Calculate similarities}
%----------------------------------
\label{sec:similarities}
We decided to use a vector model for calculating the similarities between
patient cases and therapy chapters. For each document, both patient cases
and therapy chapters, we created a vector with all the terms and their
TF-IDF value. These document-vectors gives us a pseudo term-document matrix,
without having to create an actual matrix. As there were over 30,000 terms
and almost 1000 documents, a full term-document matrix would have 30M cells.
Our pseudo term-document only have 124,065 term-weight pairs.

We have tested four different varieties of TF-IDF. For TF we tested log
normalization (\( 1 + \log f_{i,j} \)) and raw frequency (\( f_{i,j} \)).
For IDF we tested inverse frequency (\( \log(\frac{N}{n_{i}}) \)) and
probabilistic inverse frequency (\( \log(\frac{N - n_{i}}{n_{i}}) \)).
\begin{description}
	\item[A] log normalization and inverse frequency as seen in \autoref{eq:tfidf}
	\item[B] log normalization and probabilistic inverse frequency
	\item[C] raw frequency and inverse frequency 
	\item[D] raw frequency and probabilistic inverse frequency
\end{description}
\( f_{i,j} \) is the frequency of term \( i \) in document \( j \), \( N \)
is the total number of documents, while \( n_{i} \) is the document frequency
of term \( i \).
\begin{equation} \label{eq:tfidf}
	w_{i,j} =
	\begin{cases}
		(1 + \log f_{i,j}) \times \log(\frac{N}{n_{i}}) & \text{if } f_{i,j} > 0 \\
		0												& \text{otherwise}
	\end{cases}
\end{equation}

For each clinical note (which we have defined to be a patient case), we
calculated the similarities with all therapy chapter vectors. The similarity is
calculated as the \emph{cosine of the angle} between the two vectors, as seen
in \autoref{eq:sim}. \( d_{j} \) is the term vector of therapy chapter \( j \),
\( q \) is a patient case vector and \( t \) is the total number of terms.
\begin{align} \label{eq:sim}
	sim(d_{j}, q) &= \frac{\vec{d_{j}} \bullet \vec{q}}{|\vec{d}| \times |\vec{q}|} \nonumber \\
				  &= \frac{\sum_{i=1}^{t} w_{i,j} \times w_{i,q}}{\sqrt{\sum_{i=1}^{t} w_{i,j}^2} \times \sqrt{\sum_{i=1}^{t} w_{i,q}^2}}
\end{align}
We sort results based on their similarity score (highest first), and return
the first 10 results.


%===============
\section{Result}
%===============
\label{sec:result}
The results of task 3 with TF-IDF variety A (\autoref{eq:tfidf}) can be seen in
\autoref{tab:task3a} and \autoref{tab:task3b} in \autoref{appendix}.
Case in the tables are patient case/clinical note number and rank is the rank
of the relevant therapy chapter.


%===================
\section{Evaluation}
%===================
\label{sec:evaluation}
Evaluating results for a given query give great insight in what level of quality and correctness you can expect from an information retrieval system. Doing this evaluation requires human relevance judgement.

The importance of the evaluation can be seen in the context of the mandatory assignment; if a doctor were to base treatment of a patient on the results of our information retrieval system, it is crucial to know the reliability of the results. A consequence of incorrect results could lead to wrong treatment and even death.


\subsection{Automatic evaluation}
%--------------------------------
Every good information retrieval system require evaluation of the results. This is a complex task, which is difficult and time consuming for humans to do. It would be beneficial to be able to automatically evaluate the results.

The problem with automatic evaluation is with deciding if a retrieved document
is relevant or not. As the queries in this task is patient cases, which is
known beforehand, we can create a criteria for whether a document is relevant
or not. We have simply taken all the words in the patient cases and created a
list of those with medical importance, and if both the query and a retrieved
document contains such a word we deem it relevant. These medical terms are
listed in \autoref{tab:medicalterms}.

An example of terms shared between each retrieved therapy
chapter and patient case one can be seen in \autoref{tab:terms}, where
relevant medical terms are boldfaced. The patient case concerns a patient with
diabetes mellitus and the first result ``T3.1 Diabetes mellitus'' is spot on,
while the second result ``T20.2.1. Bronkial astma'' is not relevant at all.

We have calculated average precision at ten documents seen (P@10) and
R-precision. In the first patient case six out of ten retrieved documents were
automatically deemed relevant, giving a precision of \( 60\% \) and
R-precision of \( 0.67 \) (as there are four relevant documents in top six).
\autoref{tab:precision} list P@10 and R-precision for all eight patient cases
with the four different TF-IDF definitions A to D as described in
\autoref{sec:similarities}.

\begin{table}[tbp] \footnotesize \center
\caption{Terms shared between patient case 1 and relevant chapters\label{tab:terms}}
\begin{tabularx}{\textwidth}{c l l c X}
    \toprule
    Rank & Chapter & Score & Relevant & Terms \\
    \midrule
	1 & T3.1 & 0.0832 & Yes & bruker, delvis, henvisning, hatt, \textbf{acetonlukt}, injeksjon, år, hurtigvirkende, hvert, \textbf{mellitus}, siste, lite, hurtig, \textbf{insulin}, håndtere, synes, flere, dessuten, vurderer, \textbf{diabetes}, kvelden, måltid, langtidsvirkende, \textbf{blodtrykk}, normalt, døgn, sykehus \\
	2 & T10.2.1 & 0.0429 & No & bruker, delvis, hurtig, flere, dessuten, vurderer \\
	3 & T14.5.1 & 0.0407 & Yes & \textbf{insulin}, uteblir \\
	4 & T23.1.1.2 & 0.0372 & Yes & lite, \textbf{insulin}, dessuten, normalt, døgn \\
	5 & T5.4.1 & 0.0372 & Yes & delvis, år, fått, hvert, lite, håndtere, synes, flere, \textbf{diabetes} \\
	6 & T14.2.1 & 0.0332 & No & bruker, siste, brukt, flere, tatt \\
	7 & T10.2.1.1 & 0.0325 & No & bruker, delvis, henvisning, hatt, år, fått, hvert, hurtig, synes, flere, langtidsvirkende, døgn \\
	8 & T18.1.4 & 0.0309 & No & hvert, kontroller \\
	9 & T16.13.1 & 0.0304 & Yes & tørr, huden, \textbf{mellitus}, flere, \textbf{diabetes} \\
	10 & *T9.1.5 & 0.0290 & Yes & \textbf{injiserer}, hatt, injeksjon, år, hurtig, \textbf{blodtrykk}, sykehus \\
	\bottomrule
\end{tabularx}
\end{table}

\begin{table}[tbp] \footnotesize \center
\caption{Precision for each patient case\label{tab:precision}}
\begin{tabular}{c c c c c c c c c}
    \toprule
    & \multicolumn{4}{c}{Precision @ 10} & \multicolumn{4}{c}{R-precision} \\
	\cmidrule(r){2-9}
	Case & A & B & \textbf{C} & D & A & B & \textbf{C} & \textbf{D} \\
    \midrule
	1 & 60\% & 60\% & 70\% & 50\% & 0.67 & 0.67 & 0.71 & 0.60 \\
	2 & 50\% & 60\% & 70\% & 60\% & 0.80 & 0.67 & 0.71 & 0.83 \\
	3 & 90\% & 80\% & 80\% & 80\% & 0.89 & 0.88 & 0.75 & 0.75 \\
	4 & 70\% & 70\% & 60\% & 70\% & 0.71 & 0.86 & 0.83 & 0.86 \\
	5 & 80\% & 80\% & 90\% & 90\% & 0.88 & 0.88 & 0.89 & 0.89 \\
	6 & 80\% & 80\% & 80\% & 80\% & 0.75 & 0.75 & 0.88 & 0.88 \\
	7 & 100\% & 100\% & 100\% & 100\% & 1.00 & 1.00 & 1.00 & 1.00 \\
	8 & 90\% & 90\% & 80\% & 80\% & 0.89 & 0.89 & 0.88 & 0.88 \\
    \midrule
	Avg & 77.5\% & 77.5\% & \textbf{78.8\%} & 76.2\% & 0.82 & 0.82 & \textbf{0.83} & \textbf{0.83} \\
	\bottomrule
\end{tabular}
\end{table}

\subsubsection{Kendal tau coefficient}
Rank correlation metrics are an automatic way for comparing two ranking
methods, to determine how differently one varies from another. It does not
consider relevance of retrieved documents, only the relative ordering of two
rankings.

We have calculated the Kendall tau coefficient for the four ranking methods
described in \autoref{sec:similarities}, which can be found in
\autoref{tab:kendalltau}. When calculating precision we limit results to the
top ten, but for Kendall tau coefficient we have taken all returned results.
\begin{table}[tbp] \footnotesize \center
\caption{Kendall tau coefficients\label{tab:kendalltau}}
\begin{tabular}{c c c c c c c}
    \toprule
	Case & A vs B & A vs C & A vs D & B vs C & B vs D & C vs D \\
    \midrule
	1 & 0.981 & 0.941 & 0.947 & 0.931 & 0.944 & 0.978 \\
	2 & 0.984 & 0.937 & 0.940 & 0.931 & 0.939 & 0.981 \\
	3 & 0.989 & 0.948 & 0.950 & 0.945 & 0.950 & 0.988 \\
	4 & 0.975 & 0.957 & 0.959 & 0.942 & 0.962 & 0.971 \\
	5 & 0.985 & 0.951 & 0.952 & 0.945 & 0.953 & 0.983 \\
	6 & 0.971 & 0.954 & 0.954 & 0.937 & 0.960 & 0.964 \\
	7 & 0.981 & 0.944 & 0.944 & 0.936 & 0.946 & 0.978 \\
	8 & 0.985 & 0.943 & 0.948 & 0.935 & 0.944 & 0.984 \\
    \midrule
	Avg & 0.981 & 0.947 & 0.949 & 0.938 & 0.950 & 0.979 \\
	\bottomrule
\end{tabular}
\end{table}

\subsubsection{Summary}
Kendall tau coefficients shows there is little difference between the four
varities for calculating TF-IDF, which is supported by the small differences
in average precisions. For these specific patient cases we can conclude that
method C (raw frequency and inverse frequency) gives best results.

\subsection{Manual evaluation}
%-----------------------------
To see that our automatic evaluation method is sound, we have done a manual evaluation of the results. We are not experts in the medical field, but when reading a case and the suggested result, we can easily determine if the result is relevant (at least to a certain degree). 

After a closer look at the 10 suggested results for each case, we see that most of the suggested therapy chapters were relevant. Only one case retrieves low amount of relevant therapy chapters, but by looking at the R-precision we can tell that the chapters that actually were relevant must have been among the top results. Probably result 1, 2 and 3. For the other cases, we see the same tendency; R-precision values are high. This suggests that our algorithm for finding the relevant results is good.

It is important to consider how many therapy chapters that actually can be relevant for a case. It might only exist three relevant chapters for case 2, thereby the rest of the results are not relevant. In such cases it is important to have the relevant chapters as top results, which our system seems to handle well. 


%=====================
% Section: Appendices
%=====================
\appendix
\section{Appendix}
\label{appendix}
This appendix list results from task 3, and stopwords and medical terms.
The results are listed in \autoref{tab:task3a} and \autoref{tab:task3b}.
Stopwords removed from patient cases and therapy chapters are listed in
\autoref{tab:stopwords}.
Medical terms found in patient cases used for evaluating relevant results are
listed in \autoref{tab:medicalterms}.

\begin{table}[htbp] \footnotesize \center
\caption{Task 3 results (A)\label{tab:task3a}}
\begin{tabular}{c c l}
    \toprule
    Case & Rank & Relevant chapter \\
    \midrule
	1 & 1 & T3.1: Diabetes mellitus \\
	 & 2 & T10.2.1: Bronkial astma \\
	 & 3 & T14.5.1: Polycystisk ovarialt syndrom (PCOS) \\
	 & 4 & T23.1.1.2: Faste og stress \\
	 & 5 & T5.4.1: Schizofreni \\
	 & 6 & T14.2.1: Forskyvning av normal menstruasjon \\
	 & 7 & T10.2.1.1: Mild og moderat astma \\
	 & 8 & T18.1.4: Kontroll og oppfølging \\
	 & 9 & T16.13.1: Generalisert kløe \\
	 & 10 & *T9.1.5: Anafylaktoide reaksjoner \\
	\addlinespace
	2 & 1 & T10.2: Obstruktiv lungesykdom \\
	 & 2 & T10.2.2: Kronisk obstruktiv lungesykdom (kols) \\
	 & 3 & T8.4.1.2.2: Atrioventrikulær nodal reentrytakykardi (nodal takykardi) \\
	 & 4 & T10.2.1: Bronkial astma \\
	 & 5 & T8.3.2.2: Hjerteinfarkt med ST-elevasjon \\
	 & 6 & T3.1: Diabetes mellitus \\
	 & 7 & T10.8: Sarkoidose \\
	 & 8 & T15.3.7: Liten melkeproduksjon \\
	 & 9 & T5.3.1.3: Alkohol abstinensreaksjoner \\
	 & 10 & T6.2.2: Klasehodepine («Cluster headache», Hortons hodepine) \\
	\addlinespace
	3 & 1 & *T1.10: Akutt bakteriell meningitt \\
	 & 2 & T3.1: Diabetes mellitus \\
	 & 3 & T8.1: Hypertensjon \\
	 & 4 & T1.11: Bakteriell endokarditt \\
	 & 5 & T16.7.1: Skabb \\
	 & 6 & T8.2.1: Malign hypertensjon \\
	 & 7 & T19.1: Feber \\
	 & 8 & T8.2.2: Hypertensjonsencefalopati \\
	 & 9 & T8.3.2.2: Hjerteinfarkt med ST-elevasjon \\
	 & 10 & T14.6.4: Akutt bekkeninfeksjon \\
	\addlinespace
	4 & 1 & T8.3: Koronarsykdom \\
	 & 2 & T11.1.1.4.7: Emosjonell rhinitt \\
	 & 3 & T8.2.4: Hypertensjonskrise og hjerteinfarkt eller ustabil angina \\
	 & 4 & T4.6.3: Arteriell trombose \\
	 & 5 & T8.3.1: Stabil koronarsykdom (stabil angina pectoris) \\
	 & 6 & T8.4.1.2: Paroksystisk supraventrikulær takykardi \\
	 & 7 & T10.2.2: Kronisk obstruktiv lungesykdom (kols) \\
	 & 8 & *T8.3.2.1: Ustabil angina/hjerteinfarkt uten ST-elevasjon \\
	 & 9 & T24.2.1.7: Myokardscintigrafi \\
	 & 10 & T15.3.7: Liten melkeproduksjon \\
	\bottomrule
\end{tabular}
\end{table}

\begin{table}[htbp] \footnotesize \center
\caption{Task 3 results (B)\label{tab:task3b}}
\begin{tabularx}{\textwidth}{c c X}
    \toprule
    Case & Rank & Relevant chapter \\
    \midrule
	5 & 1 & T12.10.1: Hemoroider \\
	 & 2 & T12.9.3: Dyschezi (rektumobstipasjon) \\
	 & 3 & T4.1: Anemier \\
	 & 4 & T1.6.2.1: Clostridium difficile enterokolitt \\
	 & 5 & T12.10.3: Fissura ani \\
	 & 6 & T12.11: Familiær adenomatøs polypose \\
	 & 7 & T5.5: Depresjoner \\
	 & 8 & T15.1.5: Svangerskapsindusert hypertensjon \\
	 & 9 & T4.1.3.2: Talassemi \\
	 & 10 & T13.2.5: Nevrogene blæreforstyrrelser \\
	\addlinespace
	6 & 1 & T2.2.5.1: Cancer i nyreparenkym og binyre \\
	 & 2 & T11.3.2.2: Kronisk tonsillitt \\
	 & 3 & T11.3.1.2: Kronisk faryngitt \\
	 & 4 & T1.7.7: Lymfogranuloma venereum \\
	 & 5 & T11.4.4: Halitosis \\
	 & 6 & T1.1.8: Skarlagensfeber \\
	 & 7 & T11.3.2.1: Akutt tonsillitt \\
	 & 8 & T1.6.1: Ikke-inflammatoriske, toksinpregete enteritter \\
	 & 9 & T10.3.4: Pneumonier, bakterielle og med ukjent etiologi \\
	 & 10 & T10.2.1.1: Mild og moderat astma \\
	\addlinespace
	7 & 1 & T6.2.3: Spenningshodepine (Tensjonshodepine) \\
	 & 2 & T20.2.1: Akutte smerter \\
	 & 3 & T21.1.1.2: Nevropatiske smerter \\
	 & 4 & T20.2.3.1: Praktisk gjennomføring av smertebehandling hos pasienter med kort livsprognose \\
	 & 5 & T22.4.1.1: Postoperativ grunnanalgesi \\
	 & 6 & T20.1.2.2: Opioidanalgetika \\
	 & 7 & T20.2.2.1: Praktisk gjennomføring av smertebehandling hos pasienter med antatt normal levetid \\
	 & 8 & T21.1.1.1: Nociseptive smerter \\
	 & 9 & T20.2.3.2: Bruk av sterkere opioider hos pasienter med kort livsprognose \\
	 & 10 & T6.5.1: Multippel sklerose \\
	\addlinespace
	8 & 1 & T11.3.2.1: Akutt tonsillitt \\
	 & 2 & T1.1.8: Skarlagensfeber \\
	 & 3 & T1.7.5: Syfilis \\
	 & 4 & T11.3.1.1: Akutt faryngitt \\
	 & 5 & T1.3: Mononukleose \\
	 & 6 & *T1.10: Akutt bakteriell meningitt \\
	 & 7 & T16.5.1: Pyodermier \\
	 & 8 & T11.1.2.1: Akutt rhinosinusitt \\
	 & 9 & T10.3.4: Pneumonier, bakterielle og med ukjent etiologi \\
	 & 10 & T1.11: Bakteriell endokarditt \\
	\bottomrule
\end{tabularx}
\end{table}

\begin{table}[htbp] \footnotesize \center
\caption{Norwegian stopwords\label{tab:stopwords}}
\begin{tabular}{l l l l l l}
    \toprule
    A - D & D - H & H - K & K - N & N - S & S - Å \\
    \midrule
    alle & dit & har & kun & noe & so \\
    andre & ditt & hennar & kunne & noen & som \\
    at & du & henne & kva & noka & somme \\
    av & dykk & hennes & kvar & noko & somt \\
    bare & dykkar & her & kvarhelst & nokon & start \\
    begge & då & hit & kven & nokor & stille \\
    behandling & eg & hjå & kvi & nokre & syk \\
    ble & ein & ho & kvifor & ny & så \\
    blei & eit & hoe & lage & nå & sånn \\
    bli & eitt & honom & lang & når & tid \\
    blir & eller & hos & lege & og & til \\
    blitt & elles & hoss & lik & også & tilbake \\
    bort & en & hossen & like & om & um \\
    bra & ene & hun & man & opp & under \\
    bruk & eneste & hva & mange & oss & upp \\
    bruke & enhver & hvem & me & over & ut \\
    både & enn & hver & med & pasienten & uten \\
    båe & er & hvilke & medan & pasienter & var \\
    bør & et & hvilken & meg & pga & vart \\
    ca & ett & hvis & meget & på & varte \\
    da & etter & hvor & mellom & rett & ved \\
    de & for & hvordan & men & riktig & verdi \\
    deg & fordi & hvorfor & mens & samme & vere \\
    dei & forsøke & i & mer & seg & verte \\
    deim & fra & ikke & mest & selv & vi \\
    deira & fram & ikkje & mg & si & vil \\
    deires & få & ingen & mi & sia & ville \\
    dem & før & ingi & min & sidan & vite \\
    den & først & inkje & mine & siden & vore \\
    denne & gjorde & inn & mitt & sin & vors \\
    der & gjøre & innen & mot & sine & vort \\
    dere & god & inni & mye & sist & vår \\
    deres & gå & ja & mykje & sitt & være \\
    det & går & jeg & må & sjøl & vært \\
    dette & ha & kan & måte & skal & å \\
    di & hadde & kom & ned & skulle &  \\
    din & han & korleis & nei & slik &  \\
    disse & hans & korso & no & slutt &  \\
    \bottomrule
\end{tabular}
\end{table}

\begin{table}[htbp] \footnotesize \center
\caption{Medical terms\label{tab:medicalterms}}
\begin{tabular}{l l l}
    \toprule
    A - H & H - P & P - V \\
    \midrule
    acetonlukt & hemoglobin & penicillin \\
    allergier & hemolytiske & penicillintabletter \\
    analgetika & hemorroider & peroral \\
    angina & hjernehinnebetennelse & petekkier \\
    antibiotika & hjertelydene & pharynx \\
    apocillin & hodepine & postoperativt \\
    astma & hostet & pulmicort \\
    atrovent & influensa & pulsen \\
    attføring & injiserer & pusteplager \\
    avføring & insulin & rektaleksplorasjon \\
    avføringen & insulinkrevende & respirator \\
    bakteriologisk & insulinpenn & røntgenbildet \\
    blod & intramuskulære & sacrum \\
    blodkulturer & intravenøs & sederende \\
    blodmangel & intravenøse & serogruppe \\
    blodprøver & intravenøst & serumspeil \\
    blodprøves & intuberes & skjelett \\
    blodtrykk & ketorax & slim \\
    blodtrykket & kloramfenikol & smertelindring \\
    blødningskilde & kneplager & småblødninger \\
    bricanyl & kortisonpreparat & spinalpunksjon \\
    bronkiene & krampeanfall & spinalvæske \\
    brystsmerter & kvalm & spinalvæsken \\
    cancer & lungeflatene & spirometri \\
    colli & lymfeknuter & spirometriundersøkelse \\
    colonoscopi & mellitus & stetoskop \\
    desorientert & meningitidis & strept.test \\
    diabetes & meniskoperert & streptokokker \\
    diazepam & meniskplager & streptokokktonsilitt \\
    diplokokker & metastaser & støtdoser \\
    dolcontin & mikrobiologisk & submandibulært \\
    ekspiratoriske & morfin & sukkersyke \\
    ekspiriet & muskulatur & svelgbesvær \\
    endetarmsåpningen & muskulært & tarmveggen \\
    endoskopisk & nakkestiv & tonsilleforstørrelse \\
    femoris & napren & tonsillene \\
    foetor & neisseria & tonsiller \\
    fractura & nevrogen & totalprotese \\
    gane & nitroglycerin & tykktarmen \\
    glandler & opioider & vaksinasjon \\
    halebenet & ortopedisk & væsketillførsel \\
    halsflora & palpasjonsømhet &  \\
    halsinfeksjon & paralgin &  \\
    halsprøve & pectoris &  \\
    \bottomrule
\end{tabular}
\end{table}


\end{document}

