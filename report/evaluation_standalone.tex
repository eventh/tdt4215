<<<<<<< HEAD
\documentclass[a4paper, 12pt]{article}
\usepackage[T1]{fontenc}
\usepackage[utf8]{inputenc}
\usepackage[english]{babel}
\usepackage{textcomp} % Euro symbols etc
\usepackage{graphicx} % support graphics
\usepackage{hyperref} % links in the document
\usepackage{float} % position of figures
\usepackage{paralist} % inline lists
\usepackage{verbatim} % multi-line comments
\usepackage{listings} % Syntax colored code
\usepackage{booktabs} % Professional tables
\usepackage{tabularx} % Simple column stretching
\usepackage{multirow} % Row spanning
\usepackage{wrapfig} % Wrap text around figures
\usepackage{array}
\usepackage[normalsize, bf]{caption}
\usepackage{color}
\usepackage{textcomp}
\usepackage{fixltx2e}

% Configure links in pdfs
\hypersetup{
    bookmarksopen=false, % Hide bookmarks menu
    colorlinks=true, % Don't wrap links in colored boxes
%    pdfborder={0 0 0} % Remove ugly boxes
}


%============
% Top matter
%============
\title{TDT4215 Web-intelligence\\Project task 4: evaluation}
\author{Group 1:\\Even Wiik Thomassen, Terje Snarby, Weilin Wang}
\date{\today}

\begin{document}

%\begin{abstract}
%Your abstract goes here
%\end{abstract}

\maketitle
\tableofcontents
%\newpage


%=====================
\section{Introduction}
%=====================
This paper presents task 4 of the project in TDT4215 Web-intelligence at
Norwegian University of Science and Technology (NTNU). It will
describe the method we used to solve task 3 (\autoref{sec:method}),
the results from task 3 (\autoref{sec:result}),
and evaluation of the results (\autoref{sec:evaluation}).


%===============
\section{Method}
%===============
\label{sec:method}
This section describes what methods we used to solve the given task.

\subsection{Preprocessing and parsing}
%-------------------------------------
Patient cases were provided as a Word file, which included eight cases.
We created a text file for each case, and made sure they were utf-8.



\subsubsection{Stopwords}


\subsection{}
%-------------


%===============
\section{Result}
%===============
\label{sec:result}


%===================
\section{Evaluation}
%===================
\label{sec:evaluation}
Evaluation of results are important, in order to achieve high quality and correctness for a given query.

\subsection{Automatic Evaluation}
Given the importance of result evaluation, it are to be a part of a good information retrieval system. On the other hand, having humans, especially experts, to do this job would be a costly matter. To circumvent this problem, automatic solutions to do the job have been developed.

In the following subsections, we describe methods and techniques that we could have utilized to make automatic evaluation of our search results.



%=====================
% Section: Appendices
%=====================
\appendix
\section{Appendices}
This section list patient cases used as input in this project, with Norwegian
stopwords removed. The stopwords are listed in \autoref{tab:stopwords}.

\chapter{Stop words}
\autoref{tab:stopwords} contains a list of Norwegian stop words used on search
queries such as patient cases and therapy chapters.
An initial list were % TODO: add a reference to
Additional words with low relevenance, but which are frequently used in
patient cases, have been added.

\begin{table}[htbp] \footnotesize \center
\caption{Norwegian stop words\label{tab:stopwords}}
\begin{tabular}{l l l l l l}
    \toprule
    A - D & D - H & H - K & K - N & N - S & S - Å \\
    \midrule
    alle & ditt & har & korso & no & so \\
    andre & du & hennar & kun & noe & som \\
    at & dykk & henne & kunne & noen & somme \\
    av & dykkar & hennes & kva & noka & somt \\
    bare & då & her & kvar & noko & start \\
    begge & eg & hit & kvarhelst & nokon & stille \\
    ble & ein & hjå & kven & nokor & syk \\
    blei & eit & ho & kvi & nokre & så \\
    bli & eitt & hoe & kvifor & ny & sånn \\
    blir & eller & honom & lage & nå & tid \\
    blitt & elles & hos & lang & når & til \\
    bort & en & hoss & lege & og & tilbake \\
    bra & ene & hossen & lik & også & um \\
    bruk & eneste & hun & like & om & under \\
    bruke & enhver & hva & man & opp & upp \\
    både & enn & hvem & mange & oss & ut \\
    båe & er & hver & me & over & uten \\
    ca & et & hvilke & med & pga & var \\
    da & ett & hvilken & medan & på & vart \\
    de & etter & hvis & meg & rett & varte \\
    deg & for & hvor & meget & riktig & ved \\
    dei & fordi & hvordan & mellom & samme & verdi \\
    deim & forsøke & hvorfor & men & seg & vere \\
    deira & fra & i & mens & selv & verte \\
    deires & fram & ikke & mer & si & vi \\
    dem & få & ikkje & mest & sia & vil \\
    den & før & ingen & mi & sidan & ville \\
    denne & først & ingi & min & siden & vite \\
    der & gjorde & inkje & mine & sin & vore \\
    dere & gjøre & inn & mitt & sine & vors \\
    deres & god & innen & mot & sist & vort \\
    det & gå & inni & mye & sitt & vår \\
    dette & går & ja & mykje & sjøl & være \\
    di & ha & jeg & må & skal & vært \\
    din & hadde & kan & måte & skulle & å \\
    disse & han & kom & ned & slik & én \\
    dit & hans & korleis & nei & slutt &  \\
    \bottomrule
\end{tabular}
\end{table}



\begin{table}[htbp] \footnotesize \center
\caption[Patient case 1 to 8]{Patient case 1\label{tab:case1}}
\begin{tabularx}{\textwidth}{c X}
    \toprule
    \# & Lines (stopwords removed) \\
    \midrule
    1 & Eva Andersen skoleelev hatt insulinkrevende diabetes mellitus 3 år \\
    2 & bror diabetes brukt insulin flere år \\
    3 & bruker insulinpenn hurtigvirkende insulin injiserer huden hvert måltid dessuten injeksjon langtidsvirkende insulin kvelden \\
    4 & lite motivert håndtere sukkersyke \\
    5 & uteblir kontroller, delvis tatt insulin periodevis ignorert kostråd \\
    6 & synes leit fått sukkersyke \\
    7 & Eva siste døgn følt tungpusten, kvalm kastet opp \\
    8 & Rødmusset \\
    9 & Hurtig pust \\
    10 & Tørr pannen \\
    11 & Acetonlukt \\
    12 & Normalt blodtrykk \\
    13 & delvis uklar, vurderer henvisning sykehus \\
    \bottomrule
\end{tabularx}
\end{table}

\begin{table}[htbp] \footnotesize \center
\caption[]{Patient case 2\label{tab:case2}}
\begin{tabularx}{\textwidth}{c X}
    \toprule
    \# & Lines (stopwords removed) \\
    \midrule
    1 & 56 år gammel mann stort sett frisk tidligere bortsett meniskplager ført meniskoperert knær \\
    2 & arbeidet avløser jordbruk/skogbruk \\
    3 & attføring kneplager \\
    4 & Lungepoliklinikken hatt tungpust siste 3 - 4 månedene \\
    5 & tungpust hvile, tung pusten bakker trapper , stoppe hvile gått 400 -- 500 meter flat mark, kortere distanser \\
    6 & tillegg tungpust hatt følelsen tett brystet \\
    7 & hostet mye, fått del slim grønt farge føler "der noe" brystet \\
    8 & lagt merke "skrål piping" brystet \\
    9 & symptomene hatt omtrent lenge følt verre pusten \\
    10 & nærmere utspørring kommer nok tyngre pusten siste par år \\
    11 & merket fått problemer holde følge jevnaldringer situasjoner tidligere gått greitt festet spesielt brakt bane \\
    12 & Imidlertid inntrådt merkbar forverring pusten siste 3 -- 4 måneder \\
    13 & aldri plaget astma allergier tidligere livet \\
    14 & røkt ca. 40 år, mesteparten tiden 3 pakker rulletobakk uka, reduserte pakke uka par år sluttet helt 3 måneder pusteplager \\
    15 & undersøkelsen Lungeavdelingen ubesværet pusten hvile \\
    16 & lytting stetoskop lungeflatene høres unormale lyder ("pipelyder") puster (i ekspiriet), mindre grad puster inn \\
    17 & Hjertelydene normale \\
    18 & Pulsen 80 regelmessig \\
    19 & Røntgenbildet lungene normalt \\
    20 & gjort spirometri viser VK (vitalkapasiteten volumet luft maksimalt fylle lungene pust) 2,56 liter (53\% normalt) FEV1 (det forserte ekspiratoriske volum volumet blåse lungene sekund) 1,28 liter (34\% normalt) \\
    21 & gitt antibiotika behandlingen bricanyl Atrovent (som utvider bronkiene ) pulmicort (et kortisonpreparat inhalerer) hvert betydelig bedre idet pusten lettere, hosten bedre grønnfargete oppspyttet forsvant gjenvant vante tilstand \\
    22 & spirometriundersøkelse viste måned VK 3,20 liter (66\% normalt) FEV1 2,02 liter (53\% normalt), altså betydelig bedring \\
    \bottomrule
\end{tabularx}
\end{table}

\begin{table}[htbp] \footnotesize \center
\caption[]{Patient case 3\label{tab:case3}}
\begin{tabularx}{\textwidth}{c X}
    \toprule
    \# & Lines (stopwords removed) \\
    \midrule
    1 & Hanne 9. klasse ungdomsskolen, to yngre søsken \\
    2 & tidligere betydning, aldri hospitalisert \\
    3 & morgen februar klager Hanne føler syk, litt vondt halsen, hodepine smerter hele kroppen temperatur 38,7 °C \\
    4 & del influensa distriktet tror foreldrene Hanne holder utvikle \\
    5 & holder hjemme skolen dårligere utover dagen \\
    6 & far kommer hjem 16-tiden, Hanne litt omtåket, ligger døs temperaturen måles 40,9 °C \\
    7 & undersøkelsen finner legen desorientert, nakkestiv spredt kroppen petekkier ("småblødninger" huden) \\
    8 & Blodtrykket måles 105/70 mm Hg \\
    9 & Hanne innlegges umiddelbart Barneklinikken mistanke "smittsom hjernehinnebetennelse" \\
    10 & mottagelsen sykehuset Hanne undersøkt vakthavende lege, henges intravenøs væsketillførsel, blodprøves takes gjøres spinalpunksjon \\
    11 & Spinalvæsken tydelig blakket \\
    12 & 2 blodkulturer tatt, startes umiddelbart intravenøs antibiotika Penicillin G Kloramfenikol \\
    13 & par timer innkomsten får Hanne par minutters krampeanfall, behandles diazepam intravenøst \\
    14 & besluttes Hanne intuberes legges respirator \\
    15 & Dagen innkomst rapporterer mikrobiologisk laboratorium funn Gram-negative diplokokker spinalvæske blodkulturer, kommer oppvekst Neisseria meningitidis serogruppe C \\
    16 & Hannes to yngre søsken får penicillintabletter uke (Helsedirektoratets forskrift) \\
    17 & Kommunelegen setter igang vaksinasjon hele Hannes familie, nære venner skoleklasse \\
    \bottomrule
\end{tabularx}
\end{table}

\begin{table}[htbp] \footnotesize \center
\caption[]{Patient case 4\label{tab:case4}}
\begin{tabularx}{\textwidth}{c X}
    \toprule
    \# & Lines (stopwords removed) \\
    \midrule
    1 & Trond Øvrebotten, 42 år, oppsøker fastlegen 2-3 måneder følt ubehag, trykk brystet anstrengelse \\
    2 & gang kjent opphisselse \\
    3 & røyker 10-15 sigaretter daglig, trener spiser sunn mat \\
    4 & far døde plutselig ca. 50 år gammel \\
    5 & Legen starter medikamentell pasientens symptomer angina pectoris henviser spesialistundersøkelse \\
    6 & rekker komme innlegges sykehuset grunn kraftige brystsmerter litt bedre nitroglycerin, helt bort \\
    \bottomrule
\end{tabularx}
\end{table}

\begin{table}[htbp] \footnotesize \center
\caption[]{Patient case 5\label{tab:case5}}
\begin{tabularx}{\textwidth}{c X}
    \toprule
    \# & Lines (stopwords removed) \\
    \midrule
    1 & 64 år gammel kvinne tidligere stort sett frisk \\
    2 & siste 5 månedene merket følelse ufullstendig tømming avføring \\
    3 & flere anledninger spor friskt blod avføringen \\
    4 & tilskriver hemorroider hatt før, ventet over \\
    5 & egen finner galt vanlig undersøkelse \\
    6 & rektaleksplorasjon (kjenne endetarmsåpningen finger) kjenner legen vidt kanten uregelmessighet tarmveggen \\
    7 & Legen ser spor gamle ytre hemorroider sannsynlig blødningskilde nå \\
    8 & Prøver blod avføringen positive \\
    9 & Blodprøver viser lett blodmangel (hemoglobin 10.8; normalt kjønn alder >12) \\
    10 & Øvrige blodprøver normale \\
    11 & henvist colonoscopi (endoskopisk undersøkelse tykktarmen) \\
    \bottomrule
\end{tabularx}
\end{table}

\begin{table}[htbp] \footnotesize \center
\caption[]{Patient case 6\label{tab:case6}}
\begin{tabularx}{\textwidth}{c X}
    \toprule
    \# & Lines (stopwords removed) \\
    \midrule
    1 & første konsultasjon funnet forstørrede tonsiller hvitlig belegg forstørrede glandler sider halsen \\
    2 & tatt halsprøve strept.test positiv \\
    3 & fremkom kjæresten nylig gjennomgått streptokokktonsilitt \\
    4 & startet peroral penicillin (Apocillin) vanlig dosering \\
    5 & kommer konsultasjon manglende effekt behandlingen penicillin \\
    6 & verre, fikk svelgbesvær, fikk fast føde, besværligheter væske \\
    7 & Samtidig vedvarende slapp dårlig allmenntilstand \\
    \bottomrule
\end{tabularx}
\end{table}

\begin{table}[htbp] \footnotesize \center
\caption[]{Patient case 7\label{tab:case7}}
\begin{tabularx}{\textwidth}{c X}
    \toprule
    \# & Lines (stopwords removed) \\
    \midrule
    1 & Berta fikk påvist cancer mamma våren fem år siden \\
    2 & påvist metastaser skjelett innlagt ortopedisk avdeling fractura colli femoris behandlet innsetting totalprotese \\
    3 & hensyn postoperativt forløp opptrening hoften beskrives minste problem \\
    4 & hensyn smerter sier uttalt halebenet \\
    5 & smerter høyere opp, tilsvarende os sacrum \\
    6 & stemmer godt ferske funn rtg. bekken \\
    7 & Smertene føles trykkende karakter \\
    8 & utstrålende smerter tegn overfølsomhet overliggende hudområde \\
    9 & spesiell palpasjonsømhet muskulatur \\
    10 & Således holdepunkter nevrogen muskulært betingede smerter \\
    11 & svært tilbakeholden forbruk analgetika \\
    12 & Bruker Napren E 250 x 2 samt Paracet behov \\
    13 & enig fortsetter Napren E uendret dose \\
    14 & angir intoleranse Paralgin Forte, unngår moderat virkende opioider gir heller foreløpig Ketorax 5 behov \\
    15 & Forklarer lav terskel ta Ketorax \\
    16 & justere enkeltdosene 1/2-2 tabletter, føler gir effekt uakseptable bivirkninger \\
    17 & Legger vekt smertelindring gi vesentlig økt livskvalitet \\
    18 & tidligere prøvd morfin, svært kvalm dette \\
    19 & Sannsynligvis tolerere «fast serumspeil» morfin form Dolcontin; tross kvalm intramuskulære intravenøse støtdoser \\
    20 & orientert forholdsvis kort utvikler toleranse sederende virkning opioider \\
    \bottomrule
\end{tabularx}
\end{table}

\begin{table}[htbp] \footnotesize \center
\caption[]{Patient case 8\label{tab:case8}}
\begin{tabularx}{\textwidth}{c X}
    \toprule
    \# & Lines (stopwords removed) \\
    \midrule
    1 & Thomas 8. klasse aktiv gutt fotball volleyball \\
    2 & par dager vondt halsen, dårlig allmenntilstand feber tilkaller foreldrene legevakt \\
    3 & Foreldrene frykter Thomas halsinfeksjon lurer antibiotika (Thomas reise leirskole forbindelse konfirmasjons-forberedelsene 3 dager) \\
    4 & undersøkelse pharynx finner legen tonsillene forstørrete, ses hvitlig belegg tonsillene \\
    5 & hovne lymfeknuter submandibulært \\
    6 & Legen besluttet starte penicillin-behandling (tabletter Apocillin 0,5 + 0,5 + 1 mill IE) tatt halsprøve innsendes bakteriologisk dyrking \\
    7 & Tredje dag legevaktbesøk Thomas fortsatt omkring 39°C kommer undersøkelse legekontoret \\
    8 & undersøkelse finnes økende tonsilleforstørrelse - møtes nesten midtlinjen \\
    9 & Tonsillene belagt, ses flere petekkier (småblødninger) bløte gane \\
    10 & stygg foetor ex ore \\
    11 & puster gjennom munnen tett nesen \\
    12 & Telefonhenvendelse mikrobiologisk laboratorium avklarer innsendte halsprøve viser moderat hemolytiske streptokokker gr C blandet halsflora \\
    \bottomrule
\end{tabularx}
\end{table}



\end{document}

=======
\documentclass[a4paper, 12pt]{article}
\usepackage[T1]{fontenc}
\usepackage[utf8]{inputenc}
\usepackage[english]{babel}
\usepackage{textcomp} % Euro symbols etc
\usepackage{graphicx} % support graphics
\usepackage{hyperref} % links in the document
\usepackage{float} % position of figures
\usepackage{paralist} % inline lists
\usepackage{verbatim} % multi-line comments
\usepackage{listings} % Syntax colored code
\usepackage{booktabs} % Professional tables
\usepackage{tabularx} % Simple column stretching
\usepackage{multirow} % Row spanning
\usepackage{wrapfig} % Wrap text around figures
\usepackage{array}
\usepackage[normalsize, bf]{caption}
\usepackage{color}
\usepackage{textcomp}
\usepackage{fixltx2e}

\setcounter{tocdepth}{2} % Depth of table of contents

% Configure links in pdfs
\hypersetup{
    bookmarksopen=false, % Hide bookmarks menu
    colorlinks=true, % Don't wrap links in colored boxes
%    pdfborder={0 0 0} % Remove ugly boxes
}


%============
% Top matter
%============
\title{TDT4215 Web-intelligence\\Project task 4: evaluation}
\author{Group 1:\\Even Wiik Thomassen, Terje Snarby, Weilin Wang}
\date{\today}

\begin{document}

%\begin{abstract}
%Your abstract goes here
%\end{abstract}

\maketitle
\tableofcontents
%\newpage


%=====================
\section{Introduction}
%=====================
This paper presents task 4 of the project in TDT4215 Web-intelligence at
Norwegian University of Science and Technology (NTNU). It will
describe the method we used to solve task 3 (\autoref{sec:method}),
the results from task 3 (\autoref{sec:result}),
and evaluation of the results (\autoref{sec:evaluation}).


%===============
\section{Method}
%===============
\label{sec:method}

\subsection{Preprocessing and parsing}
%-------------------------------------
Patient cases were provided as a Word file, which included eight cases. We
created a text file for each case, and made sure they were in utf-8 charset.

Therapy chapters from Norsk legemiddelhåndbok were provided as HTML files,
invalid html5 files in iso-8859-1 charset. We first preprocessed these files
by removing some of the HTML tags to make them easier to parse, and we
converted them to utf-8 charset.

We created a custom parser for parsing therapy chapters, based on Python's
HTMLParser. We parse one HTML file at a time, creating Therapy objects for
each chapter or sub*-chapter. The text found in these chapters are stored
on the objects. We stored links as a list on each object, while we preserved
their text in the object text. Sections which list relevant drugs we removed
from the text but stored the links in case they might be used later.

We store the results after preprocessing and parsing in JSON files, so they
can be quickly loaded.

\subsection{Stopwords}
%---------------------
TODO: Something about stopwords.

The list of stopwords can be found in \autoref{tab:stopwords},
and patient cases with stopwords removed can be found in \autoref{appendix}.

\subsection{Task 3: Calculate similarities}
%------------------------------------------
We decided to use a vector model for calculating the similarities between
patient cases and therapy chapters. For each document, both patient cases
and therapy chapters, we created a vector with all the terms and their
TF-IDF value. These document-vectors gives us a pseudo term-document matrix,
without having to create an actual matrix. As there were over 30,000 terms
and almost 1000 documents, a full term-document matrix would have 30M fields.

TODO: show the formulas for IDF, TF and TF-IDF..

For each clinical note (which we have defined to be a patient case), we
calculate the similarities with all therapy chapter vectors.
TODO: describe and add the formula to the sim method.

We sort the list of results based on their similarity score, and return the
top 10 results.


%===============
\section{Result}
%===============
\label{sec:result}


%===================
\section{Evaluation}
%===================
\label{sec:evaluation}


%=====================
% Section: Appendices
%=====================
\appendix
\section{Appendix}
\label{appendix}
This section list patient cases used as input in this project, with Norwegian
stopwords removed. The stopwords are listed in \autoref{tab:stopwords}.

\chapter{Stop words}
\autoref{tab:stopwords} contains a list of Norwegian stop words used on search
queries such as patient cases and therapy chapters.
An initial list were % TODO: add a reference to
Additional words with low relevenance, but which are frequently used in
patient cases, have been added.

\begin{table}[htbp] \footnotesize \center
\caption{Norwegian stop words\label{tab:stopwords}}
\begin{tabular}{l l l l l l}
    \toprule
    A - D & D - H & H - K & K - N & N - S & S - Å \\
    \midrule
    alle & ditt & har & korso & no & so \\
    andre & du & hennar & kun & noe & som \\
    at & dykk & henne & kunne & noen & somme \\
    av & dykkar & hennes & kva & noka & somt \\
    bare & då & her & kvar & noko & start \\
    begge & eg & hit & kvarhelst & nokon & stille \\
    ble & ein & hjå & kven & nokor & syk \\
    blei & eit & ho & kvi & nokre & så \\
    bli & eitt & hoe & kvifor & ny & sånn \\
    blir & eller & honom & lage & nå & tid \\
    blitt & elles & hos & lang & når & til \\
    bort & en & hoss & lege & og & tilbake \\
    bra & ene & hossen & lik & også & um \\
    bruk & eneste & hun & like & om & under \\
    bruke & enhver & hva & man & opp & upp \\
    både & enn & hvem & mange & oss & ut \\
    båe & er & hver & me & over & uten \\
    ca & et & hvilke & med & pga & var \\
    da & ett & hvilken & medan & på & vart \\
    de & etter & hvis & meg & rett & varte \\
    deg & for & hvor & meget & riktig & ved \\
    dei & fordi & hvordan & mellom & samme & verdi \\
    deim & forsøke & hvorfor & men & seg & vere \\
    deira & fra & i & mens & selv & verte \\
    deires & fram & ikke & mer & si & vi \\
    dem & få & ikkje & mest & sia & vil \\
    den & før & ingen & mi & sidan & ville \\
    denne & først & ingi & min & siden & vite \\
    der & gjorde & inkje & mine & sin & vore \\
    dere & gjøre & inn & mitt & sine & vors \\
    deres & god & innen & mot & sist & vort \\
    det & gå & inni & mye & sitt & vår \\
    dette & går & ja & mykje & sjøl & være \\
    di & ha & jeg & må & skal & vært \\
    din & hadde & kan & måte & skulle & å \\
    disse & han & kom & ned & slik & én \\
    dit & hans & korleis & nei & slutt &  \\
    \bottomrule
\end{tabular}
\end{table}



\begin{table}[htbp] \footnotesize \center
\caption[Patient case 1 to 8]{Patient case 1\label{tab:case1}}
\begin{tabularx}{\textwidth}{c X}
    \toprule
    \# & Lines (stopwords removed) \\
    \midrule
    1 & Eva Andersen skoleelev hatt insulinkrevende diabetes mellitus 3 år \\
    2 & bror diabetes brukt insulin flere år \\
    3 & bruker insulinpenn hurtigvirkende insulin injiserer huden hvert måltid dessuten injeksjon langtidsvirkende insulin kvelden \\
    4 & lite motivert håndtere sukkersyke \\
    5 & uteblir kontroller, delvis tatt insulin periodevis ignorert kostråd \\
    6 & synes leit fått sukkersyke \\
    7 & Eva siste døgn følt tungpusten, kvalm kastet opp \\
    8 & Rødmusset \\
    9 & Hurtig pust \\
    10 & Tørr pannen \\
    11 & Acetonlukt \\
    12 & Normalt blodtrykk \\
    13 & delvis uklar, vurderer henvisning sykehus \\
    \bottomrule
\end{tabularx}
\end{table}

\begin{table}[htbp] \footnotesize \center
\caption[]{Patient case 2\label{tab:case2}}
\begin{tabularx}{\textwidth}{c X}
    \toprule
    \# & Lines (stopwords removed) \\
    \midrule
    1 & 56 år gammel mann stort sett frisk tidligere bortsett meniskplager ført meniskoperert knær \\
    2 & arbeidet avløser jordbruk/skogbruk \\
    3 & attføring kneplager \\
    4 & Lungepoliklinikken hatt tungpust siste 3 - 4 månedene \\
    5 & tungpust hvile, tung pusten bakker trapper , stoppe hvile gått 400 -- 500 meter flat mark, kortere distanser \\
    6 & tillegg tungpust hatt følelsen tett brystet \\
    7 & hostet mye, fått del slim grønt farge føler "der noe" brystet \\
    8 & lagt merke "skrål piping" brystet \\
    9 & symptomene hatt omtrent lenge følt verre pusten \\
    10 & nærmere utspørring kommer nok tyngre pusten siste par år \\
    11 & merket fått problemer holde følge jevnaldringer situasjoner tidligere gått greitt festet spesielt brakt bane \\
    12 & Imidlertid inntrådt merkbar forverring pusten siste 3 -- 4 måneder \\
    13 & aldri plaget astma allergier tidligere livet \\
    14 & røkt ca. 40 år, mesteparten tiden 3 pakker rulletobakk uka, reduserte pakke uka par år sluttet helt 3 måneder pusteplager \\
    15 & undersøkelsen Lungeavdelingen ubesværet pusten hvile \\
    16 & lytting stetoskop lungeflatene høres unormale lyder ("pipelyder") puster (i ekspiriet), mindre grad puster inn \\
    17 & Hjertelydene normale \\
    18 & Pulsen 80 regelmessig \\
    19 & Røntgenbildet lungene normalt \\
    20 & gjort spirometri viser VK (vitalkapasiteten volumet luft maksimalt fylle lungene pust) 2,56 liter (53\% normalt) FEV1 (det forserte ekspiratoriske volum volumet blåse lungene sekund) 1,28 liter (34\% normalt) \\
    21 & gitt antibiotika behandlingen bricanyl Atrovent (som utvider bronkiene ) pulmicort (et kortisonpreparat inhalerer) hvert betydelig bedre idet pusten lettere, hosten bedre grønnfargete oppspyttet forsvant gjenvant vante tilstand \\
    22 & spirometriundersøkelse viste måned VK 3,20 liter (66\% normalt) FEV1 2,02 liter (53\% normalt), altså betydelig bedring \\
    \bottomrule
\end{tabularx}
\end{table}

\begin{table}[htbp] \footnotesize \center
\caption[]{Patient case 3\label{tab:case3}}
\begin{tabularx}{\textwidth}{c X}
    \toprule
    \# & Lines (stopwords removed) \\
    \midrule
    1 & Hanne 9. klasse ungdomsskolen, to yngre søsken \\
    2 & tidligere betydning, aldri hospitalisert \\
    3 & morgen februar klager Hanne føler syk, litt vondt halsen, hodepine smerter hele kroppen temperatur 38,7 °C \\
    4 & del influensa distriktet tror foreldrene Hanne holder utvikle \\
    5 & holder hjemme skolen dårligere utover dagen \\
    6 & far kommer hjem 16-tiden, Hanne litt omtåket, ligger døs temperaturen måles 40,9 °C \\
    7 & undersøkelsen finner legen desorientert, nakkestiv spredt kroppen petekkier ("småblødninger" huden) \\
    8 & Blodtrykket måles 105/70 mm Hg \\
    9 & Hanne innlegges umiddelbart Barneklinikken mistanke "smittsom hjernehinnebetennelse" \\
    10 & mottagelsen sykehuset Hanne undersøkt vakthavende lege, henges intravenøs væsketillførsel, blodprøves takes gjøres spinalpunksjon \\
    11 & Spinalvæsken tydelig blakket \\
    12 & 2 blodkulturer tatt, startes umiddelbart intravenøs antibiotika Penicillin G Kloramfenikol \\
    13 & par timer innkomsten får Hanne par minutters krampeanfall, behandles diazepam intravenøst \\
    14 & besluttes Hanne intuberes legges respirator \\
    15 & Dagen innkomst rapporterer mikrobiologisk laboratorium funn Gram-negative diplokokker spinalvæske blodkulturer, kommer oppvekst Neisseria meningitidis serogruppe C \\
    16 & Hannes to yngre søsken får penicillintabletter uke (Helsedirektoratets forskrift) \\
    17 & Kommunelegen setter igang vaksinasjon hele Hannes familie, nære venner skoleklasse \\
    \bottomrule
\end{tabularx}
\end{table}

\begin{table}[htbp] \footnotesize \center
\caption[]{Patient case 4\label{tab:case4}}
\begin{tabularx}{\textwidth}{c X}
    \toprule
    \# & Lines (stopwords removed) \\
    \midrule
    1 & Trond Øvrebotten, 42 år, oppsøker fastlegen 2-3 måneder følt ubehag, trykk brystet anstrengelse \\
    2 & gang kjent opphisselse \\
    3 & røyker 10-15 sigaretter daglig, trener spiser sunn mat \\
    4 & far døde plutselig ca. 50 år gammel \\
    5 & Legen starter medikamentell pasientens symptomer angina pectoris henviser spesialistundersøkelse \\
    6 & rekker komme innlegges sykehuset grunn kraftige brystsmerter litt bedre nitroglycerin, helt bort \\
    \bottomrule
\end{tabularx}
\end{table}

\begin{table}[htbp] \footnotesize \center
\caption[]{Patient case 5\label{tab:case5}}
\begin{tabularx}{\textwidth}{c X}
    \toprule
    \# & Lines (stopwords removed) \\
    \midrule
    1 & 64 år gammel kvinne tidligere stort sett frisk \\
    2 & siste 5 månedene merket følelse ufullstendig tømming avføring \\
    3 & flere anledninger spor friskt blod avføringen \\
    4 & tilskriver hemorroider hatt før, ventet over \\
    5 & egen finner galt vanlig undersøkelse \\
    6 & rektaleksplorasjon (kjenne endetarmsåpningen finger) kjenner legen vidt kanten uregelmessighet tarmveggen \\
    7 & Legen ser spor gamle ytre hemorroider sannsynlig blødningskilde nå \\
    8 & Prøver blod avføringen positive \\
    9 & Blodprøver viser lett blodmangel (hemoglobin 10.8; normalt kjønn alder >12) \\
    10 & Øvrige blodprøver normale \\
    11 & henvist colonoscopi (endoskopisk undersøkelse tykktarmen) \\
    \bottomrule
\end{tabularx}
\end{table}

\begin{table}[htbp] \footnotesize \center
\caption[]{Patient case 6\label{tab:case6}}
\begin{tabularx}{\textwidth}{c X}
    \toprule
    \# & Lines (stopwords removed) \\
    \midrule
    1 & første konsultasjon funnet forstørrede tonsiller hvitlig belegg forstørrede glandler sider halsen \\
    2 & tatt halsprøve strept.test positiv \\
    3 & fremkom kjæresten nylig gjennomgått streptokokktonsilitt \\
    4 & startet peroral penicillin (Apocillin) vanlig dosering \\
    5 & kommer konsultasjon manglende effekt behandlingen penicillin \\
    6 & verre, fikk svelgbesvær, fikk fast føde, besværligheter væske \\
    7 & Samtidig vedvarende slapp dårlig allmenntilstand \\
    \bottomrule
\end{tabularx}
\end{table}

\begin{table}[htbp] \footnotesize \center
\caption[]{Patient case 7\label{tab:case7}}
\begin{tabularx}{\textwidth}{c X}
    \toprule
    \# & Lines (stopwords removed) \\
    \midrule
    1 & Berta fikk påvist cancer mamma våren fem år siden \\
    2 & påvist metastaser skjelett innlagt ortopedisk avdeling fractura colli femoris behandlet innsetting totalprotese \\
    3 & hensyn postoperativt forløp opptrening hoften beskrives minste problem \\
    4 & hensyn smerter sier uttalt halebenet \\
    5 & smerter høyere opp, tilsvarende os sacrum \\
    6 & stemmer godt ferske funn rtg. bekken \\
    7 & Smertene føles trykkende karakter \\
    8 & utstrålende smerter tegn overfølsomhet overliggende hudområde \\
    9 & spesiell palpasjonsømhet muskulatur \\
    10 & Således holdepunkter nevrogen muskulært betingede smerter \\
    11 & svært tilbakeholden forbruk analgetika \\
    12 & Bruker Napren E 250 x 2 samt Paracet behov \\
    13 & enig fortsetter Napren E uendret dose \\
    14 & angir intoleranse Paralgin Forte, unngår moderat virkende opioider gir heller foreløpig Ketorax 5 behov \\
    15 & Forklarer lav terskel ta Ketorax \\
    16 & justere enkeltdosene 1/2-2 tabletter, føler gir effekt uakseptable bivirkninger \\
    17 & Legger vekt smertelindring gi vesentlig økt livskvalitet \\
    18 & tidligere prøvd morfin, svært kvalm dette \\
    19 & Sannsynligvis tolerere «fast serumspeil» morfin form Dolcontin; tross kvalm intramuskulære intravenøse støtdoser \\
    20 & orientert forholdsvis kort utvikler toleranse sederende virkning opioider \\
    \bottomrule
\end{tabularx}
\end{table}

\begin{table}[htbp] \footnotesize \center
\caption[]{Patient case 8\label{tab:case8}}
\begin{tabularx}{\textwidth}{c X}
    \toprule
    \# & Lines (stopwords removed) \\
    \midrule
    1 & Thomas 8. klasse aktiv gutt fotball volleyball \\
    2 & par dager vondt halsen, dårlig allmenntilstand feber tilkaller foreldrene legevakt \\
    3 & Foreldrene frykter Thomas halsinfeksjon lurer antibiotika (Thomas reise leirskole forbindelse konfirmasjons-forberedelsene 3 dager) \\
    4 & undersøkelse pharynx finner legen tonsillene forstørrete, ses hvitlig belegg tonsillene \\
    5 & hovne lymfeknuter submandibulært \\
    6 & Legen besluttet starte penicillin-behandling (tabletter Apocillin 0,5 + 0,5 + 1 mill IE) tatt halsprøve innsendes bakteriologisk dyrking \\
    7 & Tredje dag legevaktbesøk Thomas fortsatt omkring 39°C kommer undersøkelse legekontoret \\
    8 & undersøkelse finnes økende tonsilleforstørrelse - møtes nesten midtlinjen \\
    9 & Tonsillene belagt, ses flere petekkier (småblødninger) bløte gane \\
    10 & stygg foetor ex ore \\
    11 & puster gjennom munnen tett nesen \\
    12 & Telefonhenvendelse mikrobiologisk laboratorium avklarer innsendte halsprøve viser moderat hemolytiske streptokokker gr C blandet halsflora \\
    \bottomrule
\end{tabularx}
\end{table}



\end{document}

>>>>>>> 739a4073d5db00d1418289159084f4ebbe93504e
