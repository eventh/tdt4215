\documentclass[11pt,a4paper]{report}
\usepackage[T1]{fontenc}
\usepackage[utf8]{inputenc}
\usepackage[english]{babel}
\usepackage{textcomp} % Euro symbols etc
\usepackage{graphicx} % support graphics
\usepackage{hyperref} % links in the document
\usepackage{float} % position of figures
\usepackage{paralist} % inline lists
\usepackage{verbatim} % multi-line comments
\usepackage{listings} % Syntax colored code
\usepackage{booktabs} % Professional tables
\usepackage{tabularx} % Simple column stretching
\usepackage{multirow} % Row spanning
\usepackage{wrapfig} % Wrap text around figures
\usepackage{amsmath} % Advanced maths
\usepackage[normalsize, bf]{caption} % Hide some tables from list of tables
\usepackage{array}
\usepackage{color}
\usepackage[Glenn]{fncychap}
\usepackage{fixltx2e}

\setcounter{tocdepth}{1} % Depth of table of contents

% Configure links in pdfs
\hypersetup{
    bookmarksopen=false, % Hide bookmarks menu
    colorlinks=true, % Don't wrap links in colored boxes
%    pdfborder={0 0 0} % Remove ugly boxes
}


%============
% Top matter
%============
\title{Group project}
\author{Even Wiik Thomassen, Terje Snarby, Weilin Wang}
\date{\today}

\begin{document}


%============
% Title page
%============
\begin{titlepage}
\begin{center}
% Upper part
\includegraphics[width=0.45\textwidth]{./img/NTNU-logo.png}\\[5cm]
\textsc{\large Department of Computer and Information Science}\\[0.2cm]
\textsc{\Large TDT4215 --- Web-Intelligence}\\[0.5cm]

% Title
\rule{\linewidth}{0.2mm} \\[0.4cm]
{ \LARGE \bfseries Project Report --- Group One}\\[0.2cm]
\rule{\linewidth}{0.2mm} \\[1.5cm]

% Author etc
\begin{minipage}{0.4\textwidth}
\begin{flushleft} \large
\emph{Authors:}\\
Even Wiik \textsc{Thomassen}\\
Terje \textsc{Snarby}\\
Weilin \textsc{Wang}
\end{flushleft}
\end{minipage}

% Bottom of the page
\vfill
{\large Word count: 5300}\\[0.2cm]
{\large \today}
\end{center}
\end{titlepage}


%==========
% Abstract
%==========
\begin{abstract}
Classifying treatment to patients can be complicated and error-prone. It could be beneficial to provide a system to health professionals that can automatically classify patient notes. We have created such a system with the help of vector space model based on TF-IDF and probabilistic model based on BM25F. This paper describes and evaluates such a system.
\end{abstract}


%===================
% Table of Contents
%===================
\pagenumbering{roman}
\clearpage
\phantomsection
\addcontentsline{toc}{chapter}{Contents}
\tableofcontents


%=================
% List of Figures
%=================
%\clearpage
%\phantomsection
%\addcontentsline{toc}{chapter}{List of figures}
%\listoffigures


%================
% List of Tables
%================
\setcounter{tocdepth}{1} % Depth of table of contents
\clearpage
\phantomsection
\addcontentsline{toc}{chapter}{List of tables}
\listoftables


%==========
% Chapters
%==========
\clearpage
\pagenumbering{arabic}
%---------------------
\chapter{Introduction}
%---------------------
This paper documents the mandatory group assignment in TDT4215 Web-intelligence
at Norwegian University of Science and Technology (NTNU).
TODO: describe the overall project goals

The paper is structured as follows:
\hyperref[cha:architecture]{Chapter \ref*{cha:architecture}} describes the
architecture of our system.
\hyperref[cha:method]{Chapter \ref*{cha:method}} explains methods used to
solve the project tasks, while \autoref{cha:result} presents results.
\hyperref[cha:discussion]{Chapter \ref*{cha:discussion}} provides a discussion
of the results, while \autoref{cha:conclusion} concludes the paper.
\hyperref[appendix]{Appendix \ref*{appendix}} list stopwords, medical terms
and patient cases.

%----------------------------
\chapter{System Architecture}
%----------------------------
\label{cha:architecture}
This chapter describes the architecture of our system. The system is build as a decision support system for practitioners at a hospital. To prevent any confusion, two roles are introduced: 
\begin{itemize}
	\item {\bf User}  Practitioner that uses the system to search for relevant therapy chapters for a given patient case.
	\item {\bf System administrator(s)}  Computer savvy personnel, in this case us, that manages the system. Doing the parsing of input files and building indexes. 
\end{itemize} 

\section{Tools}
%----------------------------
This section describes the tools we have used to solve the assignment. 

\subsection{Whoosh 2.0}
Whoosh is a library written in Python to support fast indexing and searching of text collections. The library provides high performance, multifunctional queries and support for scoring algorithms.

\subsection{Stop words}
We achieved to find a Norwegian stop word list, which we needed in order to make the indexing work as intended.
http://www.wisweb.no/999/147/33899-170.html

%----------------------------
\section{System overview}
%----------------------------
We have divided the system into three parts: Python modules, input files and output files. In the following sections, each part will be explained in detail. For an outline of the system, see figure \ref{fig:system_arch}.

\begin{figure}
	\centering
	\includegraphics[width=1.1\textwidth]{./img/system_architecture2.png}\\
	\caption{Outline of the system}
	\label{fig:system_arch}
\end{figure}

\subsection{Python modules}
%----------------------------
The system consists of four Python modules; parse.py, data.py, index.py and tasks.py. Each module has its own features and tasks, and together they form the functional core of the system. An additional Python library, Whoosh, provides the core index and search functionality. In the following paragraphs each module is described in greater detail. For a full overview of the modules with their associated classes, attributes and methods, see the system's class diagram \ref{fig:class_diag}.

\begin{figure}
	\centering
	\includegraphics[width=1.1\textwidth]{./img/class_diagram.png}\\
	\caption{Class diagram of the Python modules}
	\label{fig:class_diag}
\end{figure}

\paragraph{parse.py}
This module preprocess and converts input files to the more preferred JSON format, which gives better readability and reduces complexity for the following modules, which now only have to support one type of file. The module supports multiple file formats as input, for further information see the input files section \ref{inputfiles}. The module is especially important for stripping out all the HTML tags from the therapy chapters. 

\paragraph{data.py}
The JSON files created by the parse module, are used as input to the data module. The data module holds representation of all the data the system needs. As a basis for holding the data we have the BaseData class, which is inherited by more specific classes for the different representations. As the system loads a JSON file, it determines which representation that should be used, ICD, ATC, PatientCase or Therapy. 

\paragraph{index.py}
Main module for building and managing the indices. After the data module has created the representation and holds the data, the index module can build indexes of it. This gives the system the ability to store text and to search for terms.

\paragraph{tasks.py}
This module contains the Command Line Interface (CLI), which makes the user able to interact with the system. The module contains methods to perform and solve the different tasks of the assignment, specified by the user. Output is generated by this module, either as STDOUT print or as a JSON- or LaTeX-file. Sample output can be seen in the appendices. 


\subsection{Input files}
%----------------------------
\label{inputfiles}
The system needs to support multiple file formats to be able to preprocess and parse the files given in the assignment. The ICD-10 file is a .xml file, the ATC file is a .pro file, the Norsk Legemiddel Håndbok is HTML and the cases are .txt files. So the parse module has support for: xml, pro, htm and txt. 

\subsection{Output files}
%----------------------------
The result of the system is always presented in the command line. To be able to store the results and present them in this report, it was necessary and beneficial for us to make an output feature. The system is able to print the results to a JSON- or LaTeX-file. 



%---------------
\chapter{Method}
%---------------
\label{cha:method}
This section describes methods used to solve the project tasks.


\section{Preprocessing and parsing}
%----------------------------------
We preprocess and parse input data files before we save them to disk as JSON
files. This is to prevent being able to work on one task at a time, withouth
having to do parsing each time we want to run a task. We also store task
results as JSON files.

\subsection{ICD-10}

\subsection{ATC}

\subsection{Therapy chapters}
Therapy chapters from ``Norsk legemiddelhåndbok'' were provided as\\
HTML files, invalid html5 files in iso-8859-1 charset. We first preprocessed
these files by removing some of the HTML tags to make them easier to parse,
and we converted them to utf-8 charset.

We created a custom parser for parsing therapy chapters, based on\\
Python's HTMLParser. We parse one HTML file at a time, creating Therapy
objects for each chapter or sub*-chapter. The text found in these chapters are
stored on the objects. We stored links as a list on each object, while we
preserved their text in the object text. Sections which list relevant drugs
were removed from the text but stored as they might be useful later.

We manually removed subchapter T17.2 and T19.7 as they had no title nor
contained any text.

\subsection{Patient cases}
Patient cases were provided as a Word file, which included eight cases. We
created a text file for each case, and made sure they were in utf-8 charset.


\section{Stopwords}
%------------------
To reduce the number of terms in documents, and to remove words which provide
little or no relevant information value, we removed stopwords.
We used a list of Norwegian stopwords in both ``bokmål'' and ``nynorsk''
which we found
online\footnote{Stopword source: \url{http://www.wisweb.no/999/147/33899-170.html}},
and we added a few words ourself. A complete list of these stopwords can
be found in \autoref{tab:stopwords} in \autoref{appendix}.


%Describe first methods which are used in several tasks, like Whoosh default
%ranking method BM25F.
%\url{http://packages.python.org/Whoosh/api/scoring.html#whoosh.scoring.BM27F}

\section{Task 1: Autocoding ICD-10}
%----------------------------------


\section{Task 2: Autocoding ATC}
%-------------------------------


\section{Task 3: Ranking using vector models}
%--------------------------------------------


\section{Task 4: Evaluation}
%---------------------------


\section{Task 5: Exchange evaluations}
%-------------------------------------


\section{Task 6: Improving the ranking}
%--------------------------------------


\section{Task 7: Gold standard}
%------------------------------



%---------------
\chapter{Result}
%---------------
\label{cha:result}
This chapter presents results of the preprocessing and parsing of input files and the results from the assignment tasks. For task 1 and 2 we only present a subset of the results.

\section{Preprocessing and parsing}
%----------------------------------
The results of parsing the different input files can be seen in
\autoref{tab:objects}. Each code, chapter and case is stored in an object,
and saved to JSON files.
\begin{table}[htbp] \footnotesize \center
\caption{Parsed object counts\label{tab:objects}}
\begin{tabular}{l r}
    \toprule
    Type & Count \\
    \midrule
	ATC codes & 7945 \\
	ICD10 codes & 10521 \\
	Patient cases & 8 \\
	Therapy chapters & 917 \\
	\bottomrule
\end{tabular}
\end{table}

\autoref{tab:chapters} contains statistics showing the results of parsing
``Norsk legemiddelhåndbok'' HTML files --- therapy chapters. The statistics
lists the amount of chapters, in total and with text, for each of the
different chapter types --- from chapter to subsubsubsubchapter.
%Total amount of lines '\n': 3034
%Total amount of sentences '.': 24924
\begin{table}[htbp] \footnotesize \center
\caption{Therapy chapters statistics\label{tab:chapters}}
\begin{tabular}{l r r}
    \toprule
    Chapter type & Count & With text \\
    \midrule
	Chapter & 24 & 24 \\
	Subchapter & 153 & 104 \\
	Subsubchapter & 384 & 336 \\
	Subsubsubchapter & 329 & 326 \\
	Subsubsubsubchapter & 27 & 27 \\
    \midrule
	Total & 917 & 817 \\
	\bottomrule
\end{tabular}
\end{table}

Parsing of patient cases are summarized in \autoref{tab:cases}. Stopwords
refers to stopwords in the case text which have been removed, terms is the
number of unique terms (words) in the text.
\begin{table}[htbp] \footnotesize \center
\caption{Patient cases statistics\label{tab:cases}}
\begin{tabular}{c r r r r}
    \toprule
	Case \# & Lines & Stopwords & Terms & Medical terms \\
    \midrule
	1 & 13 & 1 & 55 & 10 \\
	2 & 22 & 1 & 169 & 24 \\
	3 & 17 & 0 & 125 & 30 \\
	4 & 6 & 1 & 49 & 4 \\
	5 & 11 & 2 & 63 & 14 \\
	6 & 7 & 0 & 41 & 9 \\
	7 & 20 & 2 & 123 & 29 \\
	8 & 12 & 0 & 90 & 19 \\
    \midrule
	Total & 108 & 7 & 715 & 125 \\
	\bottomrule
\end{tabular}
\end{table}

We store input data and results in JSON format, so we can work with them
without having to parse or produce them first. To demonstrate the
effectiveness of this method we list and compare the time (in seconds) it
takes to parse input data and loading JSON files in \autoref{tab:times}.
\begin{table}[htbp] \footnotesize \center
\caption{Comparing effectivenes of JSON\label{tab:times}}
\begin{tabular}{l r r c}
    \toprule
    Type & Parse time & JSON load time & Speedup \\
    \midrule
	ATC codes & 0.209 & 0.101 & 52\% \\
	ICD10 codes & 9.496 & 0.549 & 94\% \\
	Patient cases & 0.008 & 0.001 & 88\% \\
	Therapy chapters & 11.326 & 0.177 & 98\% \\
	\bottomrule
\end{tabular}
\end{table}


\section{Task 1: Autocoding ICD-10}
%----------------------------------
A sentence can match zero to many ICD-10 codes, but only the most specific shall be claimed as a match and presented in the results. Whoosh gives us a ranked list of the codes, and from this ranking the most specific code is selected. If there is more than one ICD-10 code that scores high and the match scores of the top results are close, the top three results are presented. If there is no match, a . is printed in the result table. This also applies to the therapy chapters and ATC-classifications. 

Autocoding of ICD-10 codes against patient case 1, 4, 5, and 6 can be seen in
\autoref{tab:task1a}, which list relevant ICD-10 codes for each sentence.
\autoref{tab:task1b} list results for task 1 B (therapy chapter T1.1.1, T5.5,
T8.9.2, and T24.2.1.7). Results for both method A and method B, described in
\autoref{sec:task1}, are listed next to each other for easy comparison.
\begin{table}[htbp] \footnotesize \center
\caption{Task 1 A, patient case 1, 4, 5, and 6\label{tab:task1a}}
\begin{tabular}{c c l l}
    \toprule
    Clinical note & Sentence & Method A & Method B \\
    \midrule
	1 & 1 & E10-E14 & E12, E10-E14, E14 \\
	 & 2 & E10-E14, E23.2 & E14 \\
	 & 3 & Y61.3 & Y61.3, Y62.3, Y60.3 \\
	 & 4 & Q02, P92.3 & P92.3 \\
	 & 5 & Z97.2, Q26.3 & Z97.2 \\
	 & 6 & . & E10-E14 \\
	 & 7 & . & . \\
	 & 8 & . & . \\
	 & 9 & O36.6 & P03.5 \\
	 & 10 & . & L85.3 \\
	 & 11 & . & . \\
	 & 12 & Z01.3 & Z34 \\
	 & 13 & Z38.1, Z38.0, Z38.3 & Y87.2 \\
	\addlinespace
	4 & 1 & O96 & M08, O96, R06.2 \\
	 & 2 & O33 & Q38.4 \\
	 & 3 & P92.4 & Z58.6 \\
	 & 4 & R96.0, N88.1, H91.2 & H91.2, M23.2 \\
	 & 5 & I20 & I20.1, I20 \\
	 & 6 & O84.0 & O84.0 \\
	\addlinespace
	5 & 1 & O26.2, N88.1 & O26.2, N91.1, N91.4 \\
	 & 2 & R15, R19.5 & R19.5, R19.4 \\
	 & 3 & Y65.0 & C77.8 \\
	 & 4 & O46.9 & . \\
	 & 5 & D83 & D83 \\
	 & 6 & S63.1 & S60.0 \\
	 & 7 & I84.3 & H60.0 \\
	 & 8 & C86.6 & R85 \\
	 & 9 & Q56.4, Q56, F70 & Q56, D57.3, M93.9 \\
	 & 10 & D80.6 & D80.6 \\
	 & 11 & Y61.4, Y62.4, Y60.4 & Y61.4, Y62.4, Y60.4 \\
	\addlinespace
	6 & 1 & R98, R59 & R98, R59.9, R59.0 \\
	 & 2 & R76.2 & R76.2 \\
	 & 3 & . & B90.9 \\
	 & 4 & D83, U80.0 & U80 \\
	 & 5 & E59, E58, E60 & U80 \\
	 & 6 & Y84.4 & P24.3 \\
	 & 7 & G81.0, G82.3, G82.0 & G81.0, G82.3, G82.0 \\
	\bottomrule
\end{tabular}
\end{table}

\begin{table}[htbp] \footnotesize \center
\caption{Task 1 B, chapter T1.1.1, T5.5, T8.9.2, and T24.2.1.7\label{tab:task1b}}
\begin{tabular}{c c l l}
    \toprule
    Chapter & Sentence & Method A & Method B \\
    \midrule
	T1.1.1 & 1 & B01, B01.9, B01.8 & B01.9, B01, B01.8 \\
	 & 2 & Z20, Z20.8, Z20.9 & A88.0, Z20, Z20.0 \\
	 & 3 & B01, B01.9, B01.8 & B01.9, B02.9, B02.8 \\
	 & 4 & C21.2 & O69.8 \\
	 & 5 & Q90.2 & C92.5 \\
	 & 6 & N92.4 & Z00.2 \\
	 & 7 & R00-R99 & R00-R99 \\
	 & 8 & P36.2 & A41.0, B95.6, G11.9 \\
	 & 9 & G11.1 & B01.2†, G11.1, G11.9 \\
	 & 10 & B01 & B01, B01.9, B01.8 \\
	\addlinespace
	T5.5 & 1 & F68.0, Z53.8, Z41.9 & R19.6 \\
	 & 2 & U00-U49, Z31.5, Q95.4 & Z90.5, U00-U49, I25.2 \\
	 & 3 & F32.8 & F33, F32.8 \\
	 & 4 & F51.2 & R41.8, R41, E23 \\
	 & 5 & F41.0, F32.3, F32.2 & F32.2, F32.3, F33.2 \\
	 & 6 & F31.3 & F33.3, F33.2 \\
	 & 7 & F31.8 & R70.0, F52, I51 \\
	 & 8 & F31.4, F31.5 & F33.3 \\
	 & 9 & Z34.9, Q97.1, F32.8 & Z34.9, O84.0 \\
	 & 10 & Z29, Z29.9 & F33.3, F31.8, Z29 \\
	 & 11 & O15.9 & Y4N \\
	 & 12 & Z55.0 & R45.4, R19.6, R46.0 \\
	 & 13 & P91.4, F20.4 & F60-F69 \\
	\addlinespace
	T8.9.2 & 1 & O42.1, O42.0 & O02.9, O02.8, G45.9 \\
	 & 2 & O96, Z00.8, I63 & O96, I63.3 \\
	 & 3 & G45.9, J46 & A50.0 \\
	 & 4 & Z00, A52.2, I48 & Z00, Z92.2, Z03 \\
	 & 5 & B20-B24 & I67, I68.0 *, I67.8 \\
	\addlinespace
	T24.2.1.7 & 1 & Z56.6 & F43 \\
	 & 2 & K59.1, O34, O35 & I22 \\
	 & 3 & I20 & I20.0, I20.1, I22 \\
	 & 4 & P08.0, Y61.3, Y62.3 & Y61.3, Y62.3, Y60.3 \\
	 & 5 & O63, P92.5, L21.1 & O63, R68.1, U00-U99 \\
	 & 6 & J46, I20, I20.0 & I20.0, I44.1 \\
	 & 7 & O42.1, O42.0 & F51.2, R96.1, F20.6 \\
	 & 8 & O42.1, O42.0, Z39.0 & T80.1, T80.2, T80 \\
	 & 9 & T32.3 & Z53 \\
	\bottomrule
\end{tabular}
\end{table}


\section{Task 2: Autocoding ATC}
%-------------------------------
Autocoding of ATC codes against patient case 1, 2, 3, 7, and 8 can be seen in
\autoref{tab:task2a}, where each sentence in patient cases are listed with
relevant ATC codes. \autoref{tab:task2b} list relevant ATC codes for therapy
chapter T1.10, T2.2.5.1, T3.1, and T6.2.3.
\begin{table}[htbp] \footnotesize \center
\caption{Task 2 A, patient case 1, 2, 3, 7, and 8\label{tab:task2a}}
\begin{tabular}{c c l c l}
    \toprule
    Case & Sentence & ATC codes & Sentence & ATC codes \\
    \midrule
	1 & 1 & A10X & 8 & . \\
	 & 2 & A10X & 9 & A10AD \\
	 & 3 & A10AE & 10 & . \\
	 & 4 & . & 11 & . \\
	 & 5 & A10AB1 & 12 & . \\
	 & 6 & . & 13 & . \\
	 & 7 & . & & \\
	\addlinespace
	2 & 1 & . & 12 & . \\
	 & 2 & . & 13 & . \\
	 & 3 & . & 14 & A10AD1 \\
	 & 4 & . & 15 & . \\
	 & 5 & N1BB1 & 16 & . \\
	 & 6 & . & 17 & . \\
	 & 7 & D8AC2 & 18 & . \\
	 & 8 & . & 19 & . \\
	 & 9 & . & 20 & V3AN5, G3AA7, G3AA12 \\
	 & 10 & . & 21 & R3BB1, R3BA2, R3AC3 \\
	 & 11 & V4CX & 22 & A10AD1 \\
	\addlinespace
	3 & 1 & C1BA & 10 & . \\
	 & 2 & . & 11 & . \\
	 & 3 & . & 12 & J1CE1, D6AX2, D10AF3 \\
	 & 4 & J7BB1 & 13 & N5BA1 \\
	 & 5 & . & 14 & . \\
	 & 6 & A10AD1 & 15 & . \\
	 & 7 & . & 16 & . \\
	 & 8 & . & 17 & . \\
	 & 9 & . & & \\
	\addlinespace
	7 & 1 & . & 11 & N2 \\
	 & 2 & V9B & 12 & N2BE1, M1AE2, M1AE2 \\
	 & 3 & . & 13 & M1AE2, M1AE2 \\
	 & 4 & . & 14 & N2AA59, N2AA59 \\
	 & 5 & . & 15 & N2AB1 \\
	 & 6 & . & 16 & B5BB \\
	 & 7 & . & 17 & V10B \\
	 & 8 & . & 18 & N2AA1 \\
	 & 9 & . & 19 & N2AA1, N2AA1 \\
	 & 10 & . & 20 & N2A, Z9OP, Z9SA \\
	\addlinespace
	8 & 1 & C1BA & 7 & V4CB \\
	 & 2 & . & 8 & V4CB \\
	 & 3 & A7AA, D1AA, G1AA & 9 & . \\
	 & 4 & V4CB & 10 & A12AA12 \\
	 & 5 & . & 11 & R1AX \\
	 & 6 & J1CE2, J1CE1 & 12 & C2N \\
	\bottomrule
\end{tabular}
\end{table}

\begin{table}[htbp] \footnotesize \center
\caption{Task 2 B, chapter T1.10, T2.2.5.1, T3.1, and T6.2.3\label{tab:task2b}}
\begin{tabular}{c c l c l}
    \toprule
    Chapter & Sentence & ATC codes & Sentence & ATC codes \\
    \midrule
	T1.10 & 1 & C5A, D5A, A1AB & 6 & J1EE1, J1RA1 \\
	 & 2 & V7, V7A & 7 & C10AC, N5BA1, M4AB \\
	 & 3 & V9D, J1EB, A7EC1 & 8 & A5AB, D5BB, D10AF \\
	 & 4 & J7BC20, A7EC1, N4AC & 9 & V4CB, V4CD, V4CG \\
	 & 5 & B1AD12, B5A, C8D & 10 & J7BC20, A7EC1 \\
	\addlinespace
	T2.2.5.1 & 1 & M4AA, L3AB1, L3AB4 & 2 & A5AB, D5BB, D10AF \\
	\addlinespace
	T3.1 & 1 & C2LG51, A10X, V3AA & 23 & V3AH \\
	 & 2 & C10AC, M4AB, H2AB & 24 & H4AA1, Z0CA, A5AB \\
	 & 3 & B5D, B5BA3, B5CX1 & 25 & A5AB, D5BB, D10AF \\
	 & 4 & A10BX2, A10BX3, Z0ET & 26 & B5BA3, B5CX1, V4CA2 \\
	 & 5 & B5BB, G3, M5B & 27 & V9DX1 \\
	 & 6 & V6DB & 28 & A14A \\
	 & 7 & A10AD1 & 29 & B5BC2, D2AE1, B5BA3 \\
	 & 8 & A10BA2, Z9MF, A10BD2 & 30 & A10AE, B5BB3, B5BB \\
	 & 9 & A10AE & 31 & V7AD \\
	 & 10 & A10BD, J7BC20, A10AD1 & 32 & A10AB1, A10AD4 \\
	 & 11 & V3AH & 33 & . \\
	 & 12 & A7EC1, N4AC & 34 & V3AH \\
	 & 13 & B5BA3, B5CX1, V4CA2 & 35 & A10X, N4AC, N7 \\
	 & 14 & A10AE & 36 & C5BA, C5AX, C5BB \\
	 & 15 & A10BD3 & 37 & Z9AC \\
	 & 16 & A10AE4, A10AE5 & 38 & Z9ST, V9G \\
	 & 17 & A10AE & 39 & A5AX, M4AC, A7EC1 \\
	 & 18 & A10AC1, N4AC & 40 & Z9A2 \\
	 & 19 & C9AA5, A10AB, A10B & 41 & C10AC, M4AB, A10BA2 \\
	 & 20 & A10AC1, A10AB & 42 & . \\
	 & 21 & J4AK, S1KX, P1A & 43 & Z9AC \\
	 & 22 & V3AH, J4AK, M9AX & 44 & J4AK, V4CB, V4CD \\
	 & & & 45 & A1AB \\
	\addlinespace
	T6.2.3 & 1 & V3AA, N4AC & 4 & V9G, V4CB, V4CD \\
	 & 2 & S3, V4CJ, A7EC1 & 5 & V3AA, N3AX12, Z9BD \\
	 & 3 & N4AC & & \\
	\bottomrule
\end{tabular}
\end{table}


\section{Task 3: Ranking using vector models}
%--------------------------------------------
Ranked lists of relevant therapy chapters for each patient case can be found
in \autoref{tab:task3a} and \autoref{tab:task3b}.
\begin{table}[htbp] \footnotesize \center
\caption{Task 3 results (part 1)\label{tab:task3a}}
\begin{tabular}{c c c l}
    \toprule
    Case & Rank & Score & Relevant chapter \\
    \midrule
    1 & 1 & 0.08 & T3.1: Diabetes mellitus \\
     & 2 & 0.04 & T10.2.1: Bronkial astma \\
     & 3 & 0.04 & T14.5.1: Polycystisk ovarialt syndrom (PCOS) \\
     & 4 & 0.04 & T23.1.1.2: Faste og stress \\
     & 5 & 0.04 & T5.4.1: Schizofreni \\
     & 6 & 0.03 & T14.2.1: Forskyvning av normal menstruasjon \\
     & 7 & 0.03 & T10.2.1.1: Mild og moderat astma \\
     & 8 & 0.03 & T18.1.4: Kontroll og oppfølging \\
     & 9 & 0.03 & T16.13.1: Generalisert kløe \\
     & 10 & 0.03 & T9.1.5: Anafylaktoide reaksjoner \\
	\addlinespace
    2 & 1 & 0.08 & T10.2: Obstruktiv lungesykdom \\
     & 2 & 0.06 & T10.2.2: Kronisk obstruktiv lungesykdom (kols) \\
     & 3 & 0.05 & T8.4.1.2.2: Atrioventrikulær nodal reentrytakykardi \\
     & 4 & 0.05 & T10.2.1: Bronkial astma \\
     & 5 & 0.04 & T8.3.2.2: Hjerteinfarkt med ST-elevasjon \\
     & 6 & 0.04 & T3.1: Diabetes mellitus \\
     & 7 & 0.04 & T10.8: Sarkoidose \\
     & 8 & 0.04 & T15.3.7: Liten melkeproduksjon \\
     & 9 & 0.03 & T5.3.1.3: Alkohol abstinensreaksjoner \\
     & 10 & 0.03 & T6.2.2: Klasehodepine («Cluster headache») \\
	\addlinespace
    3 & 1 & 0.09 & T1.10: Akutt bakteriell meningitt \\
     & 2 & 0.05 & T3.1: Diabetes mellitus \\
     & 3 & 0.05 & T8.1: Hypertensjon \\
     & 4 & 0.05 & T1.11: Bakteriell endokarditt \\
     & 5 & 0.04 & T16.7.1: Skabb \\
     & 6 & 0.04 & T8.2.1: Malign hypertensjon \\
     & 7 & 0.04 & T19.1: Feber \\
     & 8 & 0.04 & T8.2.2: Hypertensjonsencefalopati \\
     & 9 & 0.04 & T8.3.2.2: Hjerteinfarkt med ST-elevasjon \\
     & 10 & 0.04 & T14.6.4: Akutt bekkeninfeksjon \\
	\addlinespace
    4 & 1 & 0.09 & T8.3: Koronarsykdom \\
     & 2 & 0.06 & T11.1.1.4.7: Emosjonell rhinitt \\
     & 3 & 0.05 & T8.2.4: Hypertensjonskrise og hjerteinfarkt eller ustabil angina \\
     & 4 & 0.05 & T4.6.3: Arteriell trombose \\
     & 5 & 0.04 & T8.3.1: Stabil koronarsykdom (stabil angina pectoris) \\
     & 6 & 0.04 & T8.4.1.2: Paroksystisk supraventrikulær takykardi \\
     & 7 & 0.04 & T10.2.2: Kronisk obstruktiv lungesykdom (kols) \\
     & 8 & 0.04 & T8.3.2.1: Ustabil angina/hjerteinfarkt uten ST-elevasjon \\
     & 9 & 0.04 & T24.2.1.7: Myokardscintigrafi \\
     & 10 & 0.03 & T15.3.7: Liten melkeproduksjon \\
	\bottomrule
\end{tabular}
\end{table}

\begin{table}[htbp] \footnotesize \center
\caption{Task 3 results (part 2)\label{tab:task3b}}
\begin{tabularx}{\textwidth}{c c c X}
    \toprule
    Case & Rank & Score & Relevant chapter \\
    \midrule
    5 & 1 & 0.06 & T12.10.1: Hemoroider \\
     & 2 & 0.06 & T12.9.3: Dyschezi (rektumobstipasjon) \\
     & 3 & 0.05 & T4.1: Anemier \\
     & 4 & 0.05 & T1.6.2.1: Clostridium difficile enterokolitt \\
     & 5 & 0.05 & T12.10.3: Fissura ani \\
     & 6 & 0.05 & T12.11: Familiær adenomatøs polypose \\
     & 7 & 0.05 & T5.5: Depresjoner \\
     & 8 & 0.04 & T15.1.5: Svangerskapsindusert hypertensjon \\
     & 9 & 0.04 & T4.1.3.2: Talassemi \\
     & 10 & 0.04 & T13.2.5: Nevrogene blæreforstyrrelser \\
	\addlinespace
    6 & 1 & 0.06 & T2.2.5.1: Cancer i nyreparenkym og binyre \\
     & 2 & 0.05 & T11.3.2.2: Kronisk tonsillitt \\
     & 3 & 0.04 & T11.3.1.2: Kronisk faryngitt \\
     & 4 & 0.04 & T1.7.7: Lymfogranuloma venereum \\
     & 5 & 0.04 & T11.4.4: Halitosis \\
     & 6 & 0.04 & T1.1.8: Skarlagensfeber \\
     & 7 & 0.03 & T11.3.2.1: Akutt tonsillitt \\
     & 8 & 0.03 & T1.6.1: Ikke-inflammatoriske, toksinpregete enteritter \\
     & 9 & 0.03 & T10.3.4: Pneumonier, bakterielle og med ukjent etiologi \\
     & 10 & 0.03 & T10.2.1.1: Mild og moderat astma \\
	\addlinespace
    7 & 1 & 0.08 & T6.2.3: Spenningshodepine (Tensjonshodepine) \\
     & 2 & 0.07 & T20.2.1: Akutte smerter \\
     & 3 & 0.06 & T21.1.1.2: Nevropatiske smerter \\
     & 4 & 0.06 & T20.2.3.1: Praktisk gjennomføring av smertebehandling hos pasienter med kort livsprognose \\
     & 5 & 0.06 & T22.4.1.1: Postoperativ grunnanalgesi \\
     & 6 & 0.06 & T20.1.2.2: Opioidanalgetika \\
     & 7 & 0.06 & T20.2.2.1: Praktisk gjennomføring av smertebehandling hos pasienter med antatt normal levetid \\
     & 8 & 0.06 & T21.1.1.1: Nociseptive smerter \\
     & 9 & 0.05 & T20.2.3.2: Bruk av sterkere opioider hos pasienter med kort livsprognose \\
     & 10 & 0.04 & T6.5.1: Multippel sklerose \\
	\addlinespace
    8 & 1 & 0.08 & T11.3.2.1: Akutt tonsillitt \\
     & 2 & 0.06 & T1.1.8: Skarlagensfeber \\
     & 3 & 0.04 & T1.7.5: Syfilis \\
     & 4 & 0.04 & T11.3.1.1: Akutt faryngitt \\
     & 5 & 0.04 & T1.3: Mononukleose \\
     & 6 & 0.04 & T1.10: Akutt bakteriell meningitt \\
     & 7 & 0.03 & T16.5.1: Pyodermier \\
     & 8 & 0.03 & T11.1.2.1: Akutt rhinosinusitt \\
     & 9 & 0.03 & T10.3.4: Pneumonier, bakterielle og med ukjent etiologi \\
     & 10 & 0.03 & T1.11: Bakteriell endokarditt \\
	\bottomrule
\end{tabularx}
\end{table}


\section{Task 4: Evaluation}
%---------------------------
An example of terms shared between retrieved therapy
chapter and patient case can be seen in \autoref{tab:terms}, where
relevant medical terms are boldfaced. The patient case concerns a patient with
diabetes mellitus and the first result ``T3.1 Diabetes mellitus'' is spot on,
while the second result ``T20.2.1. Bronkial astma'' is not relevant at all.
We have calculated average precision at ten documents seen (P@10) and
R-precision, which are listed in \autoref{tab:precision}.

Rank correlation metrics are an automatic way for comparing two ranking
methods, to determine how differently one varies from another. It does not
consider relevance of retrieved documents, only the relative ordering of two
rankings. \autoref{tab:kendalltau} list such a rank correlation metric, called
Kendall tau coefficients.

\begin{table}[htbp] \footnotesize \center
\caption{Task 4 shared terms (patient case 1)\label{tab:terms}}
\begin{tabularx}{\textwidth}{c l l c X}
    \toprule
    Rank & Chapter & Score & Relevant & Terms \\
    \midrule
	1 & T3.1 & 0.0832 & Yes & bruker, delvis, henvisning, hatt, \textbf{acetonlukt}, injeksjon, år, hurtigvirkende, hvert, \textbf{mellitus}, siste, lite, hurtig, \textbf{insulin}, håndtere, synes, flere, dessuten, vurderer, \textbf{diabetes}, kvelden, måltid, langtidsvirkende, \textbf{blodtrykk}, normalt, døgn, sykehus \\
	2 & T10.2.1 & 0.0429 & No & bruker, delvis, hurtig, flere, dessuten, vurderer \\
	3 & T14.5.1 & 0.0407 & Yes & \textbf{insulin}, uteblir \\
	4 & T23.1.1.2 & 0.0372 & Yes & lite, \textbf{insulin}, dessuten, normalt, døgn \\
	5 & T5.4.1 & 0.0372 & Yes & delvis, år, fått, hvert, lite, håndtere, synes, flere, \textbf{diabetes} \\
	6 & T14.2.1 & 0.0332 & No & bruker, siste, brukt, flere, tatt \\
	7 & T10.2.1.1 & 0.0325 & No & bruker, delvis, henvisning, hatt, år, fått, hvert, hurtig, synes, flere, langtidsvirkende, døgn \\
	8 & T18.1.4 & 0.0309 & No & hvert, kontroller \\
	9 & T16.13.1 & 0.0304 & Yes & tørr, huden, \textbf{mellitus}, flere, \textbf{diabetes} \\
	10 & T9.1.5 & 0.0290 & Yes & \textbf{injiserer}, hatt, injeksjon, år, hurtig, \textbf{blodtrykk}, sykehus \\
	\bottomrule
\end{tabularx}
\end{table}

\begin{table}[htbp] \footnotesize \center
\caption{Task 4 precision of each patient case search\label{tab:precision}}
\begin{tabular}{c c c c c c c c c}
    \toprule
    & \multicolumn{4}{c}{Precision @ 10} & \multicolumn{4}{c}{R-precision} \\
	\cmidrule(r){2-9}
	Case & A & B & \textbf{C} & D & A & B & \textbf{C} & \textbf{D} \\
    \midrule
	1 & 60\% & 60\% & 70\% & 50\% & 0.67 & 0.67 & 0.71 & 0.60 \\
	2 & 50\% & 60\% & 70\% & 60\% & 0.80 & 0.67 & 0.71 & 0.83 \\
	3 & 90\% & 80\% & 80\% & 80\% & 0.89 & 0.88 & 0.75 & 0.75 \\
	4 & 70\% & 70\% & 60\% & 70\% & 0.71 & 0.86 & 0.83 & 0.86 \\
	5 & 80\% & 80\% & 90\% & 90\% & 0.88 & 0.88 & 0.89 & 0.89 \\
	6 & 80\% & 80\% & 80\% & 80\% & 0.75 & 0.75 & 0.88 & 0.88 \\
	7 & 100\% & 100\% & 100\% & 100\% & 1.00 & 1.00 & 1.00 & 1.00 \\
	8 & 90\% & 90\% & 80\% & 80\% & 0.89 & 0.89 & 0.88 & 0.88 \\
    \midrule
	Avg & 77.5\% & 77.5\% & \textbf{78.8\%} & 76.2\% & 0.82 & 0.82 & \textbf{0.83} & \textbf{0.83} \\
	\bottomrule
\end{tabular}
\end{table}

\begin{table}[htbp] \footnotesize \center
\caption{Task 4 Kendall tau coefficients\label{tab:kendalltau}}
\begin{tabular}{c c c c c c c}
    \toprule
	Case & A vs B & A vs C & A vs D & B vs C & B vs D & C vs D \\
    \midrule
	1 & 0.981 & 0.941 & 0.947 & 0.931 & 0.944 & 0.978 \\
	2 & 0.984 & 0.937 & 0.940 & 0.931 & 0.939 & 0.981 \\
	3 & 0.989 & 0.948 & 0.950 & 0.945 & 0.950 & 0.988 \\
	4 & 0.975 & 0.957 & 0.959 & 0.942 & 0.962 & 0.971 \\
	5 & 0.985 & 0.951 & 0.952 & 0.945 & 0.953 & 0.983 \\
	6 & 0.971 & 0.954 & 0.954 & 0.937 & 0.960 & 0.964 \\
	7 & 0.981 & 0.944 & 0.944 & 0.936 & 0.946 & 0.978 \\
	8 & 0.985 & 0.943 & 0.948 & 0.935 & 0.944 & 0.984 \\
    \midrule
	Avg & 0.981 & 0.947 & 0.949 & 0.938 & 0.950 & 0.979 \\
	\bottomrule
\end{tabular}
\end{table}


\section{Task 5: Exchange evaluations}
%-------------------------------------
We have calculated precision at ten documents retrieved and R-precision
for the groups which published their results of task 3, including our own
results. Precision at ten can be seen in \autoref{tab:task5precision} while
\autoref{tab:task5r} list R-precision. Group 2 was not included as their
results did not contain any therapy-chapters. It is important to point out
that other groups might have handled parsing of therapy-chapters differently,
which would affect greatly the content of chapters and therefor the rankings.
For example the content of a sub-chapter could be included or excluded in the
parent chapter, we use the latter option.

\begin{table}[htbp] \footnotesize \center
\caption{Task 5 precision at ten documents retrieved\label{tab:task5precision}}
\begin{tabular}{c c c c c c}
    \toprule
	Case & Group 1 & Group 3 & Group 4 & Group 5 & Group 14 \\
    \midrule
	1 & 60\% & 60\% & 40\% & 80\% & 60\% \\
	2 & 50\% & 70\% & 50\% & 80\% & 40\% \\
	3 & 90\% & 60\% & 70\% & 90\% & 30\% \\
	4 & 70\% & 30\% & 50\% & 70\% & 20\% \\
	5 & 80\% & 60\% & 50\% & 70\% & 80\% \\
	6 & 80\% & 30\% & 50\% & 80\% & 10\% \\
	7 & 100\% & 40\% & 30\% & 80\% & 50\% \\
	8 & 90\% & 50\% & 70\% & 80\% & 50\% \\
    \midrule
	Avg & 77.5\% & 50.0\% & 51.2\% & 78.8\% & 42.5\% \\
	\bottomrule
\end{tabular}
\end{table}

\begin{table}[htbp] \footnotesize \center
\caption{Task 5 R-precision\label{tab:task5r}}
\begin{tabular}{c c c c c c}
    \toprule
	Case & Group 1 & Group 3 & Group 4 & Group 5 & Group 14 \\
    \midrule
	1 & 0.67 & 0.67 & 0.50 & 0.75 & 0.67 \\
	2 & 0.80 & 0.71 & 0.60 & 0.75 & 0.25 \\
	3 & 0.89 & 0.67 & 0.71 & 0.89 & 0.33 \\
	4 & 0.71 & 1.00 & 0.60 & 0.71 & 0.00 \\
	5 & 0.88 & 0.67 & 0.40 & 0.71 & 0.75 \\
	6 & 0.75 & 0.00 & 0.80 & 0.75 & 0.00 \\
	7 & 1.00 & 0.50 & 0.00 & 0.88 & 0.20 \\
	8 & 0.89 & 0.60 & 0.86 & 0.88 & 0.20 \\
    \midrule
	Avg & 0.82 & 0.60 & 0.56 & 0.79 & 0.30 \\
	\bottomrule
\end{tabular}
\end{table}


\section{Task 6: Improving the ranking}
%--------------------------------------
Results of ranking relevant therapy chapters with only using task 1 and 2
results (task 6 A) can be seen in \autoref{tab:task6a1} and
\autoref{tab:task6a2}. These results merged with task 3 results (task 6 B)
is listed in \autoref{tab:task6b1} and \autoref{tab:task6b2}.

We have calculated precision at ten documents seen and R-precision for both
task 6 A and B, which are listed in \autoref{tab:task6eval}.
\autoref{tab:task6kendall} lists Kendall tau coefficients between the results
of the three ranking methods.

\begin{table}[htbp] \footnotesize \center
\caption{Task 6 A results (part 1)\label{tab:task6a1}}
\begin{tabular}{c c c l}
    \toprule
    Case & Rank & Score & Relevant chapter \\
    \midrule
    1 & 1 & 22.90 & T3: Endokrine sykdommer \\
     & 2 & 22.20 & T3.1: Diabetes mellitus \\
     & 3 & 13.10 & T24.2: Nukleærmedisin \\
     & 4 & 12.40 & T24.2.1: Nukleærmedisinsk diagnostikk \\
     & 5 & 10.40 & T24.2.1.10: Nyrescintigrafi \\
     & 6 & 10.40 & T24.2.1.13: Skjelettscintigrafi \\
     & 7 & 8.20 & T3.2.1: Hypersekresjonstilstander \\
     & 8 & 7.90 & T12: Mage-tarmsykdommer \\
     & 9 & 7.80 & T24.2.1.19: Okkult tumor \\
     & 10 & 7.50 & T3.2.1.3: Hypofysært betinget Cushings syndrom \\
	\addlinespace
    2 & 1 & 13.30 & T3.1: Diabetes mellitus \\
     & 2 & 9.80 & T10: Nedre luftveissykdommer \\
     & 3 & 9.70 & T1: Infeksjonssykdommer \\
     & 4 & 8.80 & T15: Graviditet, fødsel og amming \\
     & 5 & 8.80 & T24.2: Nukleærmedisin \\
     & 6 & 8.50 & T10.2: Obstruktiv lungesykdom \\
     & 7 & 8.30 & T17.1: Betennelsesaktige, revmatiske sykdommer \\
     & 8 & 8.30 & T11: Sykdommer i øvre luftveier, øre, munn og svelg \\
     & 9 & 8.20 & T14.1.1.1: Livmorinnlegg \\
     & 10 & 8.10 & T15.3: Amming \\
	\addlinespace
    3 & 1 & 9.80 & T1: Infeksjonssykdommer \\
     & 2 & 9.60 & T6: Nevrologiske sykdommer \\
     & 3 & 8.90 & T6.1: Epilepsi, feberkramper \\
     & 4 & 8.30 & T1.2: Influensa \\
     & 5 & 8.20 & T6.1.2: Feberkramper \\
     & 6 & 6.30 & T17: Muskel- og skjelettsykdommer \\
     & 7 & 6.10 & T3.1: Diabetes mellitus \\
     & 8 & 5.80 & T24.2: Nukleærmedisin \\
     & 9 & 5.60 & T17.1: Betennelsesaktige, revmatiske sykdommer \\
     & 10 & 5.10 & T24.2.1: Nukleærmedisinsk diagnostikk \\
	\addlinespace
    4 & 1 & 16.40 & T8: Hjerte- og karsykdommer \\
     & 2 & 15.30 & T8.3: Koronarsykdom \\
     & 3 & 14.60 & T8.3.1: Stabil koronarsykdom (stabil angina pectoris) \\
     & 4 & 11.30 & T8.3.2: Ustabil koronarsykdom (ustabil angina) \\%, hjerteinfarkt uten ST-elevasjon, hjerteinfarkt med ST-elevasjon) \\
     & 5 & 10.60 & T8.3.2.2: Hjerteinfarkt med ST-elevasjon \\
     & 6 & 6.90 & T24.2: Nukleærmedisin \\
     & 7 & 6.20 & T24.2.1: Nukleærmedisinsk diagnostikk \\
     & 8 & 5.70 & T3.1: Diabetes mellitus \\
     & 9 & 5.60 & T1: Infeksjonssykdommer \\
     & 10 & 5.30 & T17.1: Betennelsesaktige, revmatiske sykdommer \\
	\bottomrule
\end{tabular}
\end{table}

\begin{table}[htbp] \footnotesize \center
\caption{Task 6 A results (part 2)\label{tab:task6a2}}
\begin{tabular}{c c c l}
    \toprule
    Case & Rank & Score & Relevant chapter \\
    \midrule
    5 & 1 & 10.50 & T1: Infeksjonssykdommer \\
     & 2 & 9.00 & T1.6: Infeksiøse enteritter \\
     & 3 & 8.80 & T24.2: Nukleærmedisin \\
     & 4 & 8.20 & T1.6.2: Bakterielle, inflammatoriske enteritter \\
     & 5 & 8.10 & T24.2.1: Nukleærmedisinsk diagnostikk \\
     & 6 & 8.10 & T12: Mage-tarmsykdommer \\
     & 7 & 7.20 & T1.6.2.4: Shigellose \\
     & 8 & 7.00 & T23.3.1.1: Hypovolemisk sjokk \\
     & 9 & 6.60 & T12.10: Anorektale forstyrrelser \\
     & 10 & 6.50 & T6: Nevrologiske sykdommer \\
	\addlinespace
    6 & 1 & 4.60 & T7.9: Øyeskader \\
     & 2 & 4.10 & T1: Infeksjonssykdommer \\
     & 3 & 3.90 & T7.9.2: Perforerende skader (øye) \\
     & 4 & 3.60 & T10: Nedre luftveissykdommer \\
     & 5 & 3.50 & T11.4: Tenner, munnsykdommer og plager \\
     & 6 & 2.90 & T1.7: Seksuelt overførbare infeksjoner (Soi) \\
     & 7 & 2.80 & T10.3: Akutte infeksjoner i nedre luftveier og lunger \\
     & 8 & 2.70 & T11.4.7: Akutt nekrotiserende gingivitt \\
     & 9 & 2.40 & T4.4.1: Defekt blodplatefunksjon \\
     & 10 & 2.40 & T16.9: Kutane bivirkninger av systemiske legemidler \\
	\addlinespace
    7 & 1 & 15.30 & T8.3.2.2: Hjerteinfarkt med ST-elevasjon \\
     & 2 & 11.80 & T21.1: Lindring av smerter og andre plager i palliativ \\
     & 3 & 11.30 & T22.4: Postoperativ fase \\
     & 4 & 11.10 & T21.1.1: Smerter \\
     & 5 & 10.50 & T22.4.1: Postoperativ smertebehandling \\
     & 6 & 10.30 & T21.1.1.1: Nociseptive smerter \\
     & 7 & 9.70 & T22.4.1.3: Opioider i postoperativ smertebehandling \\
     & 8 & 9.10 & T20: Smerter \\
     & 9 & 8.50 & T15: Graviditet, fødsel og amming \\
     & 10 & 8.40 & T20.2: Akutte og kroniske smerter \\
	\addlinespace
    8 & 1 & 11.30 & T1: Infeksjonssykdommer \\
     & 2 & 9.60 & T1.5: Urinveisinfeksjoner \\
     & 3 & 9.30 & T24.2: Nukleærmedisin \\
     & 4 & 8.90 & T1.5.1: Nedre urinveisinfeksjon \\
     & 5 & 8.60 & T24.2.1: Nukleærmedisinsk diagnostikk \\
     & 6 & 7.30 & T12: Mage-tarmsykdommer \\
     & 7 & 7.00 & T24.2.1.16: Okkult bakteriell infeksjon. Inflammatorisk tarm \\
     & 8 & 6.90 & T15: Graviditet, fødsel og amming \\
     & 9 & 6.60 & T12.5: Galleveissykdommer \\
     & 10 & 6.20 & T15.3: Amming \\
	\bottomrule
\end{tabular}
\end{table}

\begin{table}[htbp] \footnotesize \center
\caption{Task 6 B results (part 1)\label{tab:task6b1}}
\begin{tabular}{c c c l}
    \toprule
    Case & Rank & Score & Relevant chapter \\
    \midrule
    1 & 1 & 38.20 & T3.1: Diabetes mellitus \\
     & 2 & 22.90 & T3: Endokrine sykdommer \\
     & 3 & 17.10 & T24.2: Nukleærmedisin \\
     & 4 & 13.50 & T12.4.2: Hemokromatose \\
     & 5 & 12.40 & T24.2.1.10: Nyrescintigrafi \\
     & 6 & 12.40 & T24.2.1.13: Skjelettscintigrafi \\
     & 7 & 12.40 & T24.2.1: Nukleærmedisinsk diagnostikk \\
     & 8 & 12.30 & T12.2.2: Kronisk pankreatitt \\
     & 9 & 11.50 & T3.2.1.3: Hypofysært betinget Cushings syndrom \\
     & 10 & 11.30 & T16.13.1: Generalisert kløe \\
	\addlinespace
    2 & 1 & 24.50 & T10.2: Obstruktiv lungesykdom \\
     & 2 & 21.30 & T3.1: Diabetes mellitus \\
     & 3 & 19.80 & T10.2.2: Kronisk obstruktiv lungesykdom (kols) \\
     & 4 & 15.00 & T15.3.7: Liten melkeproduksjon \\
     & 5 & 14.20 & T14.1.1.1: Livmorinnlegg \\
     & 6 & 13.70 & T10.2.1: Bronkial astma \\
     & 7 & 13.40 & T8.4.1.1.1: Kronisk atrieflimmer \\
     & 8 & 11.40 & T8.4.1.2.2: Atrioventrikulær nodal reentrytakykardi \\% (nodal takykardi) \\
     & 9 & 11.30 & T24.2.1.7: Myokardscintigrafi \\
     & 10 & 11.30 & T24.2.1.2: Dopamin transporter ligand scintigrafi \\
	\addlinespace
    3 & 1 & 21.00 & T1.10: Akutt bakteriell meningitt \\
     & 2 & 16.30 & T1.2: Influensa \\
     & 3 & 16.20 & T6.1.2: Feberkramper \\
     & 4 & 16.10 & T3.1: Diabetes mellitus \\
     & 5 & 12.90 & T7.8.2: Glaukom med åpen kammervinkel \\
     & 6 & 10.80 & T15.1.4: Kronisk hypertensjon og svangerskap \\
     & 7 & 10.80 & T1.7.5: Syfilis \\
     & 8 & 10.30 & T8.1: Hypertensjon \\
     & 9 & 10.20 & T1.11: Bakteriell endokarditt \\
     & 10 & 9.90 & T8.3.2.2: Hjerteinfarkt med ST-elevasjon \\
	\addlinespace
    4 & 1 & 33.30 & T8.3: Koronarsykdom \\
     & 2 & 22.60 & T8.3.1: Stabil koronarsykdom (stabil angina pectoris) \\
     & 3 & 18.40 & T8: Hjerte- og karsykdommer \\
     & 4 & 17.30 & T8.3.2: Ustabil koronarsykdom (ustabil angina) \\%, hjerteinfarkt uten ST-elevasjon, hjerteinfarkt med ST-elevasjon) \\
     & 5 & 16.60 & T8.3.2.2: Hjerteinfarkt med ST-elevasjon \\
     & 6 & 12.30 & T24.2.1.7: Myokardscintigrafi \\
     & 7 & 12.00 & T11.1.1.4.7: Emosjonell rhinitt \\
     & 8 & 11.70 & T3.1: Diabetes mellitus \\
     & 9 & 11.20 & T8.2.4: Hypertensjonskrise og hjerteinfarkt \\%eller ustabil angina \\
     & 10 & 10.10 & T4.6.3: Arteriell trombose \\
	\bottomrule
\end{tabular}
\end{table}

\begin{table}[htbp] \footnotesize \center
\caption{Task 6 B results (part 2)\label{tab:task6b2}}
\begin{tabular}{c c c l}
    \toprule
    Case & Rank & Score & Relevant chapter \\
    \midrule
    5 & 1 & 17.80 & T12.10.1: Hemoroider \\
     & 2 & 16.00 & T1.6.2.1: Clostridium difficile enterokolitt \\
     & 3 & 14.60 & T4.1: Anemier \\
     & 4 & 13.60 & T12.11: Familiær adenomatøs polypose \\
     & 5 & 13.60 & T12.10.3: Fissura ani \\
     & 6 & 13.20 & T1.6.2.4: Shigellose \\
     & 7 & 12.40 & T12.9.3: Dyschezi (rektumobstipasjon) \\
     & 8 & 12.10 & T24.2.1: Nukleærmedisinsk diagnostikk \\
     & 9 & 11.50 & T5.5: Depresjoner \\
     & 10 & 11.00 & T23.3.1.1: Hypovolemisk sjokk \\
	\addlinespace
    6 & 1 & 12.40 & T11.3.2.2: Kronisk tonsillitt \\
     & 2 & 12.00 & T2.2.5.1: Cancer i nyreparenkym og binyre \\
     & 3 & 9.00 & T1.1.8: Skarlagensfeber \\
     & 4 & 8.40 & T4.4.1: Defekt blodplatefunksjon \\
     & 5 & 8.10 & T10.3.4: Pneumonier, bakterielle og med ukjent etiologi \\
     & 6 & 8.00 & T11.4.4: Halitosis \\
     & 7 & 8.00 & T11.3.1.2: Kronisk faryngitt \\
     & 8 & 8.00 & T1.7.7: Lymfogranuloma venereum \\
     & 9 & 6.90 & T1.11: Bakteriell endokarditt \\
     & 10 & 6.40 & T16.9: Kutane bivirkninger av systemiske legemidler \\
	\addlinespace
    7 & 1 & 23.30 & T8.3.2.2: Hjerteinfarkt med ST-elevasjon \\
     & 2 & 22.30 & T21.1.1.1: Nociseptive smerter \\
     & 3 & 21.70 & T20.2.1: Akutte smerter \\
     & 4 & 17.60 & T20.2.3.2: Bruk av sterkere opioider hos pasienter \\% med kort livsprognose \\
     & 5 & 16.50 & T22.4.1: Postoperativ smertebehandling \\
     & 6 & 16.20 & T6.2.3: Spenningshodepine (Tensjonshodepine) \\
     & 7 & 15.70 & T22.4.1.3: Opioider i postoperativ smertebehandling \\
     & 8 & 15.40 & T20.2.2.1: Praktisk gjennomføring av smertebehandling \\% hos pasienter med antatt normal levetid \\
     & 9 & 15.10 & T21.1.1: Smerter \\
     & 10 & 14.40 & T20.2: Akutte og kroniske smerter \\
	\addlinespace
    8 & 1 & 19.30 & T11.3.2.1: Akutt tonsillitt \\
     & 2 & 12.90 & T1.5.1: Nedre urinveisinfeksjon \\
     & 3 & 12.90 & T1.1.8: Skarlagensfeber \\
     & 4 & 12.70 & T1.10: Akutt bakteriell meningitt \\
     & 5 & 12.50 & T1.7.5: Syfilis \\
     & 6 & 11.30 & T24.2: Nukleærmedisin \\
     & 7 & 11.30 & T1: Infeksjonssykdommer \\
     & 8 & 10.80 & T1.12: Osteomyelitt \\
     & 9 & 10.50 & T1.13: Nekrotiserende fasciitt \\
     & 10 & 9.70 & T1.7: Seksuelt overførbare infeksjoner (Soi) \\
	\bottomrule
\end{tabular}
\end{table}

\begin{table}[htbp] \footnotesize \center
\caption{Task 6 evaluations\label{tab:task6eval}}
\begin{tabular}{c c c c c c c}
    \toprule
    & \multicolumn{3}{c}{Precision @ 10} & \multicolumn{3}{c}{R-precision} \\
	\cmidrule(r){2-7}
	Case & Task 3 & Task 6 A & \textbf{Task 6 B} & Task 3 & Task 6 A & \textbf{Task 6 B} \\
    \midrule
	1 & 60\% & 50\% & 70\% & 0.67 & 0.40 & 0.57 \\
	2 & 50\% & 40\% & 70\% & 0.80 & 0.25 & 0.86 \\
	3 & 90\% & 40\% & 100\% & 0.89 & 0.25 & 1.00 \\
	4 & 70\% & 70\% & 90\% & 0.71 & 0.86 & 0.89 \\
	5 & 80\% & 50\% & 90\% & 0.88 & 0.60 & 0.89 \\
	6 & 80\% & 30\% & 90\% & 0.75 & 0.33 & 0.89 \\
	7 & 100\% & 50\% & 900\% & 1.00 & 0.60 & 1.00 \\
	8 & 90\% & 30\% & 80\% & 0.89 & 0.67 & 0.88 \\
	Avg & 77.5\% & 45.0\% & \textbf{85.0\%} & 0.82 & 0.49 & \textbf{0.87} \\
	\bottomrule
\end{tabular}
\end{table}

\begin{table}[htbp] \footnotesize \center
\caption{Task 6 Kendall tau coefficients\label{tab:task6kendall}}
\begin{tabular}{c c c c}
    \toprule
	Case & Task 3 vs Task 6 A & Task 3 vs Task 6 B & Task 6 A vs Task 6 B \\
    \midrule
	1 & 0.725 & 0.795 & 0.810 \\
	2 & 0.611 & 0.790 & 0.792 \\
	3 & 0.606 & 0.818 & 0.733 \\
	4 & 0.692 & 0.815 & 0.762 \\
	5 & 0.688 & 0.832 & 0.813 \\
	6 & 0.740 & 0.778 & 0.690 \\
	7 & 0.627 & 0.827 & 0.716 \\
	8 & 0.663 & 0.791 & 0.808 \\
    \midrule
	Avg & 0.669 & 0.806 & 0.765 \\
	\bottomrule
\end{tabular}
\end{table}



%-------------------
\chapter{Discussion}
%-------------------
\label{cha:discussion}
In this chapter we discuss task results, limitations to the assignment and look at potential improvements to the system. 


\section{Efficient and precise results}
%--------------------------------------
The goal of the system was to provide clinicians with an efficient and precise search interface, to help them make therapy recommendations. The correctness of these recommendations is crucial for the system to be trusted by the users.

\subsection{Information retrieval system}
To make our information retrieval system fulfill the necessary properties, we had to make some fundamental architectural and design decisions.
High performance had to be prioritized to make the system efficient. To achieve this we use the fast and lightweight programming language Python, and we convert all the input files to JSON format. By having files stored as JSON, the loading time is substantially reduced.

As correctness is the most crucial property, we devoted much of our time to tweak algorithms and manually check that the results corresponded to the patient cases. For the two first tasks, we use BM25F, which is a well known algorithm that uses probability to find the most relevant results. In task 3 we make use of a vector model to determine the proximity of the terms to calculate the relevance. In task 6 the probabilistic model and vector model from the prior tasks create a new function to find the most relevant results.    

By aggregating a retrieval function based on the algorithms used in task 1,2 and 3, we end up with a system that comply with the initial requirements; efficient and precise. Mathematical ranking techniques like term frequency, inverse document frequency and term proximity are core parts of the system. These are sound and well known techniques, and used correctly they should give our system a very good therapy recommendation capability.

\subsection{Task results}
To see that our results were correct during the work, we did manual evaluations of the results. We are not experts in the medical field, but when reading a case and the suggested result, we could easily determine if the result was relevant or not (at least to a certain degree). After a closer look at the 10 suggested results for each case, we saw that most of the suggested results were relevant.

For task 1 and 2, where the probabilistic method is used, the results depends purely on the summed value of each term included in the query. This approach works, and in most cases the results are relevant. The problem is that terms should not be treated equally, some terms contains more value in the task of classifying than others. For instance the terms "motivert" and "sukkersyke", taken from case 1 sentence 4, should not be thought of as equally significant to the search. We eliminate parts of this problem by removing stopwords from the patient cases before we run the classification. In a perfect world the system would recognize words and understand the meaning/value in a given case, and weight them accordingly to get the perfect results. 

For the therapy chapter classification in task 3, we used proximity of the terms to find the most relevant results. This should give more correct classifications than the probabilistic method used in task 1 and 2. Each patient case is compared to all therapy chapters to find the closest match, which is the most relevant result. The importance of individual terms are reduced, and instead all the text in a patient case is used to classify it. Making use of more information should yield better results.

After a manual evaluation of the results from task 3, we can determine the following. Only one case retrieves low amount of relevant therapy chapters, but by looking at the R-precision we can tell that the chapters that actually were relevant must have been among the top results. Probably result 1, 2 and 3. For the other cases, we see the same tendency; R-precision values are high. This suggests that our algorithm for finding the relevant results is good. 

It is important to consider how many therapy chapters that actually can be relevant for a case. It might only exist three relevant chapters for case 2, thereby the rest of the results are not relevant. In such cases it is important to have the relevant chapters as top results, which our system seems to handle well.

\subsection{System results}
The final system make use of all the methods used to solve the tasks. This is the system we have developed to support clinicians in their work. Evaluating the results from the system give great insight in what level of quality and correctness they can expect from it. If a doctor were to base treatment of a patient on the results of our system, it is crucial to know the reliability of the results. A consequence of incorrect results could lead to wrong treatment and even death.


\section{Limitations}
%--------------------
An obvious limitation to the assignment was that only approved libraries could be used to solve the tasks. This point was up for negotiation, and other libraries could be used if approved by the staff. Typically the students would start out by looking at the limited set of choices, and most probably choose one of the suggested. This limitation might have served a good cause; leading the students in to well know libraries that were sure to work for the given tasks.
We spend a significant amount of time creating the parser of therapy chapters using only standard Python libraries, while we believe we could have solved this challenge faster using a well-known Python library
html5lib\footnote{html5lib: \url{http://code.google.com/p/html5lib/}}.

Another limitation was the lack of domain knowledge. When we do manual evaluation of results to a patient case, we can not be completely sure that the results are relevant. It might look relevant to a untrained eye. A domain expert would also discover that relevant results are missing in a result set. Testing in a real environment would overcome this limitation, see \autoref{sec:potentialimp} for more details.


\section{Potential improvements}
%-------------------------------
\label{sec:potentialimp}
This section describes possible improvements to the system. Improvements to fields such as user experience, usability, performance, algorithms and results. This could be accomplished by tweaking of the system on our own or by doing in-field testing with real patients and clinicians. The latter is the most preferred one.

\subsection{Stemmer}
We did not use any stemming on the documents, as we were unaware of any known Norwegian stemmer algorithms, but we believe using a stemmer might improve the system.

\subsection{Reference collection}
A potential improvement to the system would be to use prior knowledge about patient cases as extra input to our algorithms. Provided with case and correct classification, the system could weight treatments as more probable based on history and similarity. It could well be that a patient case does not give an exact image of what the patient's problem really is, and then it would be beneficial to have a big reference base to help classifying. 

\subsection{In-field testing}
By doing a field test of the system, we would collect valuable data that could tell us: how the system would be used, how experts want it to be, see how good our results really are, is the system to slow to be used on real cases, is the user interface not intuitive and so on. Experts would get hands-on experience and could quickly determine pros and cons of the system. We believe that this is the most important point in improving the system considerably.

\subsection{More input}
If the system was deployed in a real environment, the results from prior cases could be given as extra input to the algorithms. This input could make up a probabilistic model of what that classifications that are often correct, often wrong or where uncertainties often occur. All extra information assist the system in giving the best possible results. Clinicians could record what the system determined as treatment, what they determined as treatment, and then weeks after the treatment see if it was a correct classification or not, and feed this back in to the system.

\subsection{N-grams}
Instead of using a bag of words based vector space model, a possible improvement might be to test letter-based n-grams, where each vector cell consist of for example 4 letters found next to each other in documents. Or possibly word-based n-grams like bigrams or trigrams. Such methods would have the benefit of also taking into account proximity of letters and/or words.


%-------------------
\chapter{Conclusion}
%-------------------
\label{cha:conclusion}
Through several search and retrieval tasks we have learned how algorithms and approaches determine what results we get, and how good these results are. This assignment demonstrates the importance of incorporating multiple mathematic measures to give the best set of relevant results. In addition, we have seen that our information retrieval systems works best if it is provided with several sources of information. We achieved the best result when we combined results from the tasks for the ICD-10 codes, ATC codes and therapy chapters to classify treatment for the patient cases.

A well known management adage is ``You can't manage what you don't measure''. This implies the importance of evaluating results. As described in the report, we used both manual and automatic evaluation to measure how good the results were. We quickly realized the huge benefits of doing this automatically, but we also understood the difficulties. We believe that automatic evaluation will never be as good as the evaluation done manually by an expert of the domain. 

We can conclude that the algorithms we have used are sound, and that they solve the search and retrieval of the tasks in the assignment. 


%==============
% Bibliography
%==============
%\clearpage
%\phantomsection
%\addcontentsline{toc}{chapter}{Bibliography}
%\bibliography{references}{}
%\bibliographystyle{plain}


\appendix
%=================
\chapter{Appendix}
%=================
\label{appendix}


\section{Stopwords}
%------------------
\autoref{tab:stopwords} contains a list of Norwegian stopwords used on search
queries such as patient cases and therapy chapters.
We found started with an initial list of stopwords that we found
online\footnote{Stopword source: \url{http://www.wisweb.no/999/147/33899-170.html}}.
We added additional words with low relevance that are frequently used in
patient cases.


\section{Medical terms}
%----------------------
\autoref{tab:medicalterms} list medical terms which we found in patient cases.
These are used for automatically evaluating if a search result is relevant or
not.


\section{Patient cases}
%----------------------
This chapter contains patient cases used as input in this project.
Norwegian stop words have been removed from these patient cases.



\chapter{Stop words}
\autoref{tab:stopwords} contains a list of Norwegian stop words used on search
queries such as patient cases and therapy chapters.
An initial list were % TODO: add a reference to
Additional words with low relevenance, but which are frequently used in
patient cases, have been added.

\begin{table}[htbp] \footnotesize \center
\caption{Norwegian stop words\label{tab:stopwords}}
\begin{tabular}{l l l l l l}
    \toprule
    A - D & D - H & H - K & K - N & N - S & S - Å \\
    \midrule
    alle & ditt & har & korso & no & so \\
    andre & du & hennar & kun & noe & som \\
    at & dykk & henne & kunne & noen & somme \\
    av & dykkar & hennes & kva & noka & somt \\
    bare & då & her & kvar & noko & start \\
    begge & eg & hit & kvarhelst & nokon & stille \\
    ble & ein & hjå & kven & nokor & syk \\
    blei & eit & ho & kvi & nokre & så \\
    bli & eitt & hoe & kvifor & ny & sånn \\
    blir & eller & honom & lage & nå & tid \\
    blitt & elles & hos & lang & når & til \\
    bort & en & hoss & lege & og & tilbake \\
    bra & ene & hossen & lik & også & um \\
    bruk & eneste & hun & like & om & under \\
    bruke & enhver & hva & man & opp & upp \\
    både & enn & hvem & mange & oss & ut \\
    båe & er & hver & me & over & uten \\
    ca & et & hvilke & med & pga & var \\
    da & ett & hvilken & medan & på & vart \\
    de & etter & hvis & meg & rett & varte \\
    deg & for & hvor & meget & riktig & ved \\
    dei & fordi & hvordan & mellom & samme & verdi \\
    deim & forsøke & hvorfor & men & seg & vere \\
    deira & fra & i & mens & selv & verte \\
    deires & fram & ikke & mer & si & vi \\
    dem & få & ikkje & mest & sia & vil \\
    den & før & ingen & mi & sidan & ville \\
    denne & først & ingi & min & siden & vite \\
    der & gjorde & inkje & mine & sin & vore \\
    dere & gjøre & inn & mitt & sine & vors \\
    deres & god & innen & mot & sist & vort \\
    det & gå & inni & mye & sitt & vår \\
    dette & går & ja & mykje & sjøl & være \\
    di & ha & jeg & må & skal & vært \\
    din & hadde & kan & måte & skulle & å \\
    disse & han & kom & ned & slik & én \\
    dit & hans & korleis & nei & slutt &  \\
    \bottomrule
\end{tabular}
\end{table}



\begin{table}[htbp] \footnotesize \center
\caption{Medical terms\label{tab:medicalterms}}
\begin{tabular}{l l l l}
    \toprule
    A - E & E - K & L - R & S - V \\
    \midrule
    acetonlukt & endoskopisk & lungeflatene & sacrum\\
    allergier & femoris & lymfeknuter & sederende\\
    analgetika & foetor & mellitus & serogruppe\\
    angina & fractura & meningitidis & serumspeil\\
    antibiotika & gane & meniskoperert & skjelett\\
    apocillin & glandler & meniskplager & slim\\
    astma & halebenet & metastaser & smertelindring\\
    atrovent & halsflora & mikrobiologisk & småblødninger\\
    attføring & halsinfeksjon & morfin & spinalpunksjon\\
    avføring & halsprøve & muskulatur & spinalvæske\\
    avføringen & hemoglobin & muskulært & spinalvæsken\\
    bakteriologisk & hemolytiske & nakkestiv & spirometri\\
    blod & hemorroider & napren & spirometriundersøkelse\\
    blodkulturer & hjernehinnebetennelse & neisseria & stetoskop\\
    blodmangel & hjertelydene & nevrogen & strept.test\\
    blodprøver & hodepine & nitroglycerin & streptokokker\\
    blodprøves & hostet & opioider & streptokokktonsilitt\\
    blodtrykket & influensa & ortopedisk & støtdoser\\
    blødningskilde & injiserer & palpasjonsømhet & submandibulært\\
    bricanyl & insulin & paralgin & sukkersyke\\
    bronkiene & insulinkrevende & pectoris & svelgbesvær\\
    brystsmerter & insulinpenn & penicillin & tarmveggen\\
    cancer & intramuskulære & penicillintabletter & tonsilleforstørrelse\\
    colli & intravenøs & peroral & tonsillene\\
    colonoscopi & intravenøse & petekkier & tonsiller\\
    desorientert & intravenøst & pharynx & totalprotese\\
    diabetes & intuberes & postoperativt & tykktarmen\\
    diazepam & ketorax & pulmicort & vaksinasjon\\
    diplokokker & kloramfenikol & pulsen & væsketillførsel\\
    dolcontin & kneplager & pusteplager & \\
    ekspiratoriske & kortisonpreparat & rektaleksplorasjon & \\
    ekspiriet & krampeanfall & respirator & \\
    endetarmsåpningen & kvalm & røntgenbildet & \\
    \bottomrule
\end{tabular}
\end{table}



\begin{table}[htbp] \footnotesize \center
\caption[Patient case 1 to 8]{Patient case 1\label{tab:case1}}
\begin{tabularx}{\textwidth}{c X}
    \toprule
    \# & Lines (stopwords removed) \\
    \midrule
    1 & Eva Andersen skoleelev hatt insulinkrevende diabetes mellitus 3 år \\
    2 & bror diabetes brukt insulin flere år \\
    3 & bruker insulinpenn hurtigvirkende insulin injiserer huden hvert måltid dessuten injeksjon langtidsvirkende insulin kvelden \\
    4 & lite motivert håndtere sukkersyke \\
    5 & uteblir kontroller, delvis tatt insulin periodevis ignorert kostråd \\
    6 & synes leit fått sukkersyke \\
    7 & Eva siste døgn følt tungpusten, kvalm kastet opp \\
    8 & Rødmusset \\
    9 & Hurtig pust \\
    10 & Tørr pannen \\
    11 & Acetonlukt \\
    12 & Normalt blodtrykk \\
    13 & delvis uklar, vurderer henvisning sykehus \\
    \bottomrule
\end{tabularx}
\end{table}

\begin{table}[htbp] \footnotesize \center
\caption[]{Patient case 2\label{tab:case2}}
\begin{tabularx}{\textwidth}{c X}
    \toprule
    \# & Lines (stopwords removed) \\
    \midrule
    1 & 56 år gammel mann stort sett frisk tidligere bortsett meniskplager ført meniskoperert knær \\
    2 & arbeidet avløser jordbruk/skogbruk \\
    3 & attføring kneplager \\
    4 & Lungepoliklinikken hatt tungpust siste 3 - 4 månedene \\
    5 & tungpust hvile, tung pusten bakker trapper , stoppe hvile gått 400 -- 500 meter flat mark, kortere distanser \\
    6 & tillegg tungpust hatt følelsen tett brystet \\
    7 & hostet mye, fått del slim grønt farge føler "der noe" brystet \\
    8 & lagt merke "skrål piping" brystet \\
    9 & symptomene hatt omtrent lenge følt verre pusten \\
    10 & nærmere utspørring kommer nok tyngre pusten siste par år \\
    11 & merket fått problemer holde følge jevnaldringer situasjoner tidligere gått greitt festet spesielt brakt bane \\
    12 & Imidlertid inntrådt merkbar forverring pusten siste 3 -- 4 måneder \\
    13 & aldri plaget astma allergier tidligere livet \\
    14 & røkt ca. 40 år, mesteparten tiden 3 pakker rulletobakk uka, reduserte pakke uka par år sluttet helt 3 måneder pusteplager \\
    15 & undersøkelsen Lungeavdelingen ubesværet pusten hvile \\
    16 & lytting stetoskop lungeflatene høres unormale lyder ("pipelyder") puster (i ekspiriet), mindre grad puster inn \\
    17 & Hjertelydene normale \\
    18 & Pulsen 80 regelmessig \\
    19 & Røntgenbildet lungene normalt \\
    20 & gjort spirometri viser VK (vitalkapasiteten volumet luft maksimalt fylle lungene pust) 2,56 liter (53\% normalt) FEV1 (det forserte ekspiratoriske volum volumet blåse lungene sekund) 1,28 liter (34\% normalt) \\
    21 & gitt antibiotika behandlingen bricanyl Atrovent (som utvider bronkiene ) pulmicort (et kortisonpreparat inhalerer) hvert betydelig bedre idet pusten lettere, hosten bedre grønnfargete oppspyttet forsvant gjenvant vante tilstand \\
    22 & spirometriundersøkelse viste måned VK 3,20 liter (66\% normalt) FEV1 2,02 liter (53\% normalt), altså betydelig bedring \\
    \bottomrule
\end{tabularx}
\end{table}

\begin{table}[htbp] \footnotesize \center
\caption[]{Patient case 3\label{tab:case3}}
\begin{tabularx}{\textwidth}{c X}
    \toprule
    \# & Lines (stopwords removed) \\
    \midrule
    1 & Hanne 9. klasse ungdomsskolen, to yngre søsken \\
    2 & tidligere betydning, aldri hospitalisert \\
    3 & morgen februar klager Hanne føler syk, litt vondt halsen, hodepine smerter hele kroppen temperatur 38,7 °C \\
    4 & del influensa distriktet tror foreldrene Hanne holder utvikle \\
    5 & holder hjemme skolen dårligere utover dagen \\
    6 & far kommer hjem 16-tiden, Hanne litt omtåket, ligger døs temperaturen måles 40,9 °C \\
    7 & undersøkelsen finner legen desorientert, nakkestiv spredt kroppen petekkier ("småblødninger" huden) \\
    8 & Blodtrykket måles 105/70 mm Hg \\
    9 & Hanne innlegges umiddelbart Barneklinikken mistanke "smittsom hjernehinnebetennelse" \\
    10 & mottagelsen sykehuset Hanne undersøkt vakthavende lege, henges intravenøs væsketillførsel, blodprøves takes gjøres spinalpunksjon \\
    11 & Spinalvæsken tydelig blakket \\
    12 & 2 blodkulturer tatt, startes umiddelbart intravenøs antibiotika Penicillin G Kloramfenikol \\
    13 & par timer innkomsten får Hanne par minutters krampeanfall, behandles diazepam intravenøst \\
    14 & besluttes Hanne intuberes legges respirator \\
    15 & Dagen innkomst rapporterer mikrobiologisk laboratorium funn Gram-negative diplokokker spinalvæske blodkulturer, kommer oppvekst Neisseria meningitidis serogruppe C \\
    16 & Hannes to yngre søsken får penicillintabletter uke (Helsedirektoratets forskrift) \\
    17 & Kommunelegen setter igang vaksinasjon hele Hannes familie, nære venner skoleklasse \\
    \bottomrule
\end{tabularx}
\end{table}

\begin{table}[htbp] \footnotesize \center
\caption[]{Patient case 4\label{tab:case4}}
\begin{tabularx}{\textwidth}{c X}
    \toprule
    \# & Lines (stopwords removed) \\
    \midrule
    1 & Trond Øvrebotten, 42 år, oppsøker fastlegen 2-3 måneder følt ubehag, trykk brystet anstrengelse \\
    2 & gang kjent opphisselse \\
    3 & røyker 10-15 sigaretter daglig, trener spiser sunn mat \\
    4 & far døde plutselig ca. 50 år gammel \\
    5 & Legen starter medikamentell pasientens symptomer angina pectoris henviser spesialistundersøkelse \\
    6 & rekker komme innlegges sykehuset grunn kraftige brystsmerter litt bedre nitroglycerin, helt bort \\
    \bottomrule
\end{tabularx}
\end{table}

\begin{table}[htbp] \footnotesize \center
\caption[]{Patient case 5\label{tab:case5}}
\begin{tabularx}{\textwidth}{c X}
    \toprule
    \# & Lines (stopwords removed) \\
    \midrule
    1 & 64 år gammel kvinne tidligere stort sett frisk \\
    2 & siste 5 månedene merket følelse ufullstendig tømming avføring \\
    3 & flere anledninger spor friskt blod avføringen \\
    4 & tilskriver hemorroider hatt før, ventet over \\
    5 & egen finner galt vanlig undersøkelse \\
    6 & rektaleksplorasjon (kjenne endetarmsåpningen finger) kjenner legen vidt kanten uregelmessighet tarmveggen \\
    7 & Legen ser spor gamle ytre hemorroider sannsynlig blødningskilde nå \\
    8 & Prøver blod avføringen positive \\
    9 & Blodprøver viser lett blodmangel (hemoglobin 10.8; normalt kjønn alder >12) \\
    10 & Øvrige blodprøver normale \\
    11 & henvist colonoscopi (endoskopisk undersøkelse tykktarmen) \\
    \bottomrule
\end{tabularx}
\end{table}

\begin{table}[htbp] \footnotesize \center
\caption[]{Patient case 6\label{tab:case6}}
\begin{tabularx}{\textwidth}{c X}
    \toprule
    \# & Lines (stopwords removed) \\
    \midrule
    1 & første konsultasjon funnet forstørrede tonsiller hvitlig belegg forstørrede glandler sider halsen \\
    2 & tatt halsprøve strept.test positiv \\
    3 & fremkom kjæresten nylig gjennomgått streptokokktonsilitt \\
    4 & startet peroral penicillin (Apocillin) vanlig dosering \\
    5 & kommer konsultasjon manglende effekt behandlingen penicillin \\
    6 & verre, fikk svelgbesvær, fikk fast føde, besværligheter væske \\
    7 & Samtidig vedvarende slapp dårlig allmenntilstand \\
    \bottomrule
\end{tabularx}
\end{table}

\begin{table}[htbp] \footnotesize \center
\caption[]{Patient case 7\label{tab:case7}}
\begin{tabularx}{\textwidth}{c X}
    \toprule
    \# & Lines (stopwords removed) \\
    \midrule
    1 & Berta fikk påvist cancer mamma våren fem år siden \\
    2 & påvist metastaser skjelett innlagt ortopedisk avdeling fractura colli femoris behandlet innsetting totalprotese \\
    3 & hensyn postoperativt forløp opptrening hoften beskrives minste problem \\
    4 & hensyn smerter sier uttalt halebenet \\
    5 & smerter høyere opp, tilsvarende os sacrum \\
    6 & stemmer godt ferske funn rtg. bekken \\
    7 & Smertene føles trykkende karakter \\
    8 & utstrålende smerter tegn overfølsomhet overliggende hudområde \\
    9 & spesiell palpasjonsømhet muskulatur \\
    10 & Således holdepunkter nevrogen muskulært betingede smerter \\
    11 & svært tilbakeholden forbruk analgetika \\
    12 & Bruker Napren E 250 x 2 samt Paracet behov \\
    13 & enig fortsetter Napren E uendret dose \\
    14 & angir intoleranse Paralgin Forte, unngår moderat virkende opioider gir heller foreløpig Ketorax 5 behov \\
    15 & Forklarer lav terskel ta Ketorax \\
    16 & justere enkeltdosene 1/2-2 tabletter, føler gir effekt uakseptable bivirkninger \\
    17 & Legger vekt smertelindring gi vesentlig økt livskvalitet \\
    18 & tidligere prøvd morfin, svært kvalm dette \\
    19 & Sannsynligvis tolerere «fast serumspeil» morfin form Dolcontin; tross kvalm intramuskulære intravenøse støtdoser \\
    20 & orientert forholdsvis kort utvikler toleranse sederende virkning opioider \\
    \bottomrule
\end{tabularx}
\end{table}

\begin{table}[htbp] \footnotesize \center
\caption[]{Patient case 8\label{tab:case8}}
\begin{tabularx}{\textwidth}{c X}
    \toprule
    \# & Lines (stopwords removed) \\
    \midrule
    1 & Thomas 8. klasse aktiv gutt fotball volleyball \\
    2 & par dager vondt halsen, dårlig allmenntilstand feber tilkaller foreldrene legevakt \\
    3 & Foreldrene frykter Thomas halsinfeksjon lurer antibiotika (Thomas reise leirskole forbindelse konfirmasjons-forberedelsene 3 dager) \\
    4 & undersøkelse pharynx finner legen tonsillene forstørrete, ses hvitlig belegg tonsillene \\
    5 & hovne lymfeknuter submandibulært \\
    6 & Legen besluttet starte penicillin-behandling (tabletter Apocillin 0,5 + 0,5 + 1 mill IE) tatt halsprøve innsendes bakteriologisk dyrking \\
    7 & Tredje dag legevaktbesøk Thomas fortsatt omkring 39°C kommer undersøkelse legekontoret \\
    8 & undersøkelse finnes økende tonsilleforstørrelse - møtes nesten midtlinjen \\
    9 & Tonsillene belagt, ses flere petekkier (småblødninger) bløte gane \\
    10 & stygg foetor ex ore \\
    11 & puster gjennom munnen tett nesen \\
    12 & Telefonhenvendelse mikrobiologisk laboratorium avklarer innsendte halsprøve viser moderat hemolytiske streptokokker gr C blandet halsflora \\
    \bottomrule
\end{tabularx}
\end{table}



\end{document}

