%---------------------
\chapter{Introduction}
%---------------------
This paper documents the mandatory group assignment in TDT4215 Web-intelligence
at Norwegian University of Science and Technology (NTNU).

What we do is to implement a search system that utilizes the patient record notes for the current patient and apply it as a search query to the diverse chapters of a electronic handbook that is pharmaceutical interventions (Legemiddelhåndboka), which is to the assist practitioners in efficient and precise searching  and  enhance the keywords based search interface. In other words, the aim is to realize therapy recommendation in the electronic patient system.

The paper is structured as follows:
\hyperref[cha:architecture]{Chapter \ref*{cha:architecture}} describes the
architecture of our system.
\hyperref[cha:method]{Chapter \ref*{cha:method}} explains methods used to
solve the project tasks, while \autoref{cha:result} presents results.
\hyperref[cha:discussion]{Chapter \ref*{cha:discussion}} provides a discussion
of the results, while \autoref{cha:conclusion} concludes the paper.
Stopwords, medical terms and patient cases are listed in the
\hyperref[appendix]{Appendix}.
