%---------------
\chapter{Method}
%---------------
\label{cha:method}
This section describes methods used to solve the project tasks.


\section{Preprocessing and parsing}
%----------------------------------
We preprocess and parse input data files before we save them to disk as JSON
files. This is to prevent being able to work on one task at a time, withouth
having to do parsing each time we want to run a task. We also store task
results as JSON files.

\subsection{ICD-10}

\subsection{ATC}

\subsection{Therapy chapters}
Therapy chapters from ``Norsk legemiddelhåndbok'' were provided as\\
HTML files, invalid html5 files in iso-8859-1 charset. We first preprocessed
these files by removing some of the HTML tags to make them easier to parse,
and we converted them to utf-8 charset.

We created a custom parser for parsing therapy chapters, based on\\
Python's HTMLParser. We parse one HTML file at a time, creating Therapy
objects for each chapter or sub*-chapter. The text found in these chapters are
stored on the objects. We stored links as a list on each object, while we
preserved their text in the object text. Sections which list relevant drugs
were removed from the text but stored as they might be useful later.

We manually removed subchapter T17.2 and T19.7 as they had no title nor
contained any text.

\subsection{Patient cases}
Patient cases were provided as a Word file, which included eight cases. We
created a text file for each case, and made sure they were in utf-8 charset.


\section{Stopwords}
%------------------
To reduce the number of terms in documents, and to remove words which provide
little or no relevant information value, we removed stopwords.
We used a list of Norwegian stopwords in both ``bokmål'' and ``nynorsk''
which we found
online\footnote{Stopword source: \url{http://www.wisweb.no/999/147/33899-170.html}},
and we added a few words ourself. A complete list of these stopwords can
be found in \autoref{tab:stopwords} in \autoref{appendix}.


%Describe first methods which are used in several tasks, like Whoosh default
%ranking method BM25F.
%\url{http://packages.python.org/Whoosh/api/scoring.html#whoosh.scoring.BM27F}

\section{Task 1: Autocoding ICD-10}
%----------------------------------


\section{Task 2: Autocoding ATC}
%-------------------------------


\section{Task 3: Ranking using vector models}
%--------------------------------------------


\section{Task 4: Evaluation}
%---------------------------


\section{Task 5: Exchange evaluations}
%-------------------------------------


\section{Task 6: Improving the ranking}
%--------------------------------------


\section{Task 7: Gold standard}
%------------------------------


