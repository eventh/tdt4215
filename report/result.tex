%---------------
\chapter{Result}
%---------------
\label{cha:result}
This chapter presents our results from the different project tasks, and
parsing input files. For task 1 and 2 we only present a subset of the results.


\section{Preprocessing and parsing}
%----------------------------------
The results of parsing the different input files can be seen in
\autoref{tab:objects}. Each code, chapter and case is stored in an object,
and saved to JSON files.
\begin{table}[htbp] \footnotesize \center
\caption{Parsed object counts\label{tab:objects}}
\begin{tabular}{l r}
    \toprule
    Type & Count \\
    \midrule
	ATC codes & 7945 \\
	ICD10 codes & 10521 \\
	Patient cases & 8 \\
	Therapy chapters & 917 \\
	\bottomrule
\end{tabular}
\end{table}

\autoref{tab:chapters} contains statistics showing the results of parsing
``Norsk legemiddelhåndbok'' HTML files --- therapy chapters. The statistics
lists the amount of chapters, in total and with text, for each of the
different chapter types --- from chapter to subsubsubsubchapter.
%Total amount of lines '\n': 3034
%Total amount of sentences '.': 24924
\begin{table}[htbp] \footnotesize \center
\caption{Therapy chapters statistics\label{tab:chapters}}
\begin{tabular}{l r r}
    \toprule
    Chapter type & Count & With text \\
    \midrule
	Chapter & 24 & 24 \\
	Subchapter & 153 & 104 \\
	Subsubchapter & 384 & 336 \\
	Subsubsubchapter & 329 & 326 \\
	Subsubsubsubchapter & 27 & 27 \\
    \midrule
	Total & 917 & 817 \\
	\bottomrule
\end{tabular}
\end{table}

Parsing of patient cases are summarized in \autoref{tab:cases}. Stopwords
refers to stopwords in the case text which have been removed, terms is the
number of unique terms (words) in the text.
\begin{table}[htbp] \footnotesize \center
\caption{Patient cases statistics\label{tab:cases}}
\begin{tabular}{c r r r r}
    \toprule
	Case \# & Lines & Stopwords & Terms & Medical terms \\
    \midrule
	1 & 13 & 1 & 55 & 10 \\
	2 & 22 & 1 & 169 & 24 \\
	3 & 17 & 0 & 125 & 30 \\
	4 & 6 & 1 & 49 & 4 \\
	5 & 11 & 2 & 63 & 14 \\
	6 & 7 & 0 & 41 & 9 \\
	7 & 20 & 2 & 123 & 29 \\
	8 & 12 & 0 & 90 & 19 \\
    \midrule
	Total & 108 & 7 & 715 & 125 \\
	\bottomrule
\end{tabular}
\end{table}

We store input data and results in JSON format, so we can work with them
without having to parse or produce them first. To demonstrate the
effectiveness of this method we list and compare the time (in seconds) it
takes to parse input data and loading JSON files in \autoref{tab:times}.
\begin{table}[htbp] \footnotesize \center
\caption{Comparing effectivenes of JSON\label{tab:times}}
\begin{tabular}{l r r c}
    \toprule
    Type & Parse time & JSON load time & Speedup \\
    \midrule
	ATC codes & 0.209 & 0.101 & 52\% \\
	ICD10 codes & 9.496 & 0.549 & 94\% \\
	Patient cases & 0.008 & 0.001 & 88\% \\
	Therapy chapters & 11.326 & 0.177 & 98\% \\
	\bottomrule
\end{tabular}
\end{table}


\section{Task 1: Autocoding ICD-10}
%----------------------------------
A sentence can match zero to many ICD-10 codes, but only the most specific shall be claimed as a match and presented in the results. Whoosh gives us a ranked list of the codes, and from this ranking the most specific code is selected. If there is more than one ICD-10 code that scores high and the match scores of the top results are close, the top three results are presented. If there is no match, a . is printed in the result table. 

Autocoding of ICD-10 codes against patient case 1, 4, 5, and 6 can be seen in
\autoref{tab:task1a}, which list relevant ICD-10 codes for each sentence.
\autoref{tab:task1b} list results for task 1 B (therapy chapter T1.1.1, T5.5,
T8.9.2, and T24.2.1.7). Results for both method A and method B is listed next
to each other for easy comparison.
\begin{table}[htbp] \footnotesize \center
\caption{Task 1 A, patient case 1, 4, 5, and 6\label{tab:task1a}}
\begin{tabular}{c c l l}
    \toprule
    Clinical note & Sentence & Method A & Method B \\
    \midrule
	1 & 1 & E10-E14 & E12, E10-E14, E14 \\
	 & 2 & E10-E14, E23.2 & E14 \\
	 & 3 & Y61.3 & Y61.3, Y62.3, Y60.3 \\
	 & 4 & Q02, P92.3 & P92.3 \\
	 & 5 & Z97.2, Q26.3 & Z97.2 \\
	 & 6 & . & E10-E14 \\
	 & 7 & . & . \\
	 & 8 & . & . \\
	 & 9 & O36.6 & P03.5 \\
	 & 10 & . & L85.3 \\
	 & 11 & . & . \\
	 & 12 & Z01.3 & Z34 \\
	 & 13 & Z38.1, Z38.0, Z38.3 & Y87.2 \\
	\addlinespace
	4 & 1 & O96 & M08, O96, R06.2 \\
	 & 2 & O33 & Q38.4 \\
	 & 3 & P92.4 & Z58.6 \\
	 & 4 & R96.0, N88.1, H91.2 & H91.2, M23.2 \\
	 & 5 & I20 & I20.1, I20 \\
	 & 6 & O84.0 & O84.0 \\
	\addlinespace
	5 & 1 & O26.2, N88.1 & O26.2, N91.1, N91.4 \\
	 & 2 & R15, R19.5 & R19.5, R19.4 \\
	 & 3 & Y65.0 & C77.8 \\
	 & 4 & O46.9 & . \\
	 & 5 & D83 & D83 \\
	 & 6 & S63.1 & S60.0 \\
	 & 7 & I84.3 & H60.0 \\
	 & 8 & C86.6 & R85 \\
	 & 9 & Q56.4, Q56, F70 & Q56, D57.3, M93.9 \\
	 & 10 & D80.6 & D80.6 \\
	 & 11 & Y61.4, Y62.4, Y60.4 & Y61.4, Y62.4, Y60.4 \\
	\addlinespace
	6 & 1 & R98, R59 & R98, R59.9, R59.0 \\
	 & 2 & R76.2 & R76.2 \\
	 & 3 & . & B90.9 \\
	 & 4 & D83, U80.0 & U80 \\
	 & 5 & E59, E58, E60 & U80 \\
	 & 6 & Y84.4 & P24.3 \\
	 & 7 & G81.0, G82.3, G82.0 & G81.0, G82.3, G82.0 \\
	\bottomrule
\end{tabular}
\end{table}

\begin{table}[htbp] \footnotesize \center
\caption{Task 1 B, chapter T1.1.1, T5.5, T8.9.2, and T24.2.1.7\label{tab:task1b}}
\begin{tabular}{c c l l}
    \toprule
    Chapter & Sentence & Method A & Method B \\
    \midrule
	T1.1.1 & 1 & B01, B01.9, B01.8 & B01.9, B01, B01.8 \\
	 & 2 & Z20, Z20.8, Z20.9 & A88.0, Z20, Z20.0 \\
	 & 3 & B01, B01.9, B01.8 & B01.9, B02.9, B02.8 \\
	 & 4 & C21.2 & O69.8 \\
	 & 5 & Q90.2 & C92.5 \\
	 & 6 & N92.4 & Z00.2 \\
	 & 7 & R00-R99 & R00-R99 \\
	 & 8 & P36.2 & A41.0, B95.6, G11.9 \\
	 & 9 & G11.1 & B01.2†, G11.1, G11.9 \\
	 & 10 & B01 & B01, B01.9, B01.8 \\
	\addlinespace
	T5.5 & 1 & F68.0, Z53.8, Z41.9 & R19.6 \\
	 & 2 & U00-U49, Z31.5, Q95.4 & Z90.5, U00-U49, I25.2 \\
	 & 3 & F32.8 & F33, F32.8 \\
	 & 4 & F51.2 & R41.8, R41, E23 \\
	 & 5 & F41.0, F32.3, F32.2 & F32.2, F32.3, F33.2 \\
	 & 6 & F31.3 & F33.3, F33.2 \\
	 & 7 & F31.8 & R70.0, F52, I51 \\
	 & 8 & F31.4, F31.5 & F33.3 \\
	 & 9 & Z34.9, Q97.1, F32.8 & Z34.9, O84.0 \\
	 & 10 & Z29, Z29.9 & F33.3, F31.8, Z29 \\
	 & 11 & O15.9 & Y4N \\
	 & 12 & Z55.0 & R45.4, R19.6, R46.0 \\
	 & 13 & P91.4, F20.4 & F60-F69 \\
	\addlinespace
	T8.9.2 & 1 & O42.1, O42.0 & O02.9, O02.8, G45.9 \\
	 & 2 & O96, Z00.8, I63 & O96, I63.3 \\
	 & 3 & G45.9, J46 & A50.0 \\
	 & 4 & Z00, A52.2, I48 & Z00, Z92.2, Z03 \\
	 & 5 & B20-B24 & I67, I68.0 *, I67.8 \\
	\addlinespace
	T24.2.1.7 & 1 & Z56.6 & F43 \\
	 & 2 & K59.1, O34, O35 & I22 \\
	 & 3 & I20 & I20.0, I20.1, I22 \\
	 & 4 & P08.0, Y61.3, Y62.3 & Y61.3, Y62.3, Y60.3 \\
	 & 5 & O63, P92.5, L21.1 & O63, R68.1, U00-U99 \\
	 & 6 & J46, I20, I20.0 & I20.0, I44.1 \\
	 & 7 & O42.1, O42.0 & F51.2, R96.1, F20.6 \\
	 & 8 & O42.1, O42.0, Z39.0 & T80.1, T80.2, T80 \\
	 & 9 & T32.3 & Z53 \\
	\bottomrule
\end{tabular}
\end{table}


\section{Task 2: Autocoding ATC}
%-------------------------------
Autocoding of ATC codes against patient case 1, 2, 3, 7, and 8 can be seen in
\autoref{tab:task2a}, where each sentence in patient cases are listed with
relevant ATC codes. \autoref{tab:task2b} list relevant ATC codes for therapy
chapter *T1.10, T2.2.5.1, T3.1, and T6.2.3.
\begin{table}[htbp] \footnotesize \center
\caption{Task 2 A, patient case 1, 2, 3, 7, and 8\label{tab:task2a}}
\begin{tabular}{c c l c l}
    \toprule
    Case & Sentence & ATC codes & Sentence & ATC codes \\
    \midrule
	1 & 1 & A10X & 8 & . \\
	 & 2 & A10X & 9 & A10AD \\
	 & 3 & A10AE & 10 & . \\
	 & 4 & . & 11 & . \\
	 & 5 & A10AB1 & 12 & . \\
	 & 6 & . & 13 & . \\
	 & 7 & . & & \\
	\addlinespace
	2 & 1 & . & 12 & . \\
	 & 2 & . & 13 & . \\
	 & 3 & . & 14 & A10AD1 \\
	 & 4 & . & 15 & . \\
	 & 5 & N1BB1 & 16 & . \\
	 & 6 & . & 17 & . \\
	 & 7 & D8AC2 & 18 & . \\
	 & 8 & . & 19 & . \\
	 & 9 & . & 20 & V3AN5, G3AA7, G3AA12 \\
	 & 10 & . & 21 & R3BB1, R3BA2, R3AC3 \\
	 & 11 & V4CX & 22 & A10AD1 \\
	\addlinespace
	3 & 1 & C1BA & 10 & . \\
	 & 2 & . & 11 & . \\
	 & 3 & . & 12 & J1CE1, D6AX2, D10AF3 \\
	 & 4 & J7BB1 & 13 & N5BA1 \\
	 & 5 & . & 14 & . \\
	 & 6 & A10AD1 & 15 & . \\
	 & 7 & . & 16 & . \\
	 & 8 & . & 17 & . \\
	 & 9 & . & & \\
	\addlinespace
	7 & 1 & . & 11 & N2 \\
	 & 2 & V9B & 12 & N2BE1, M1AE2, M1AE2 \\
	 & 3 & . & 13 & M1AE2, M1AE2 \\
	 & 4 & . & 14 & N2AA59, N2AA59 \\
	 & 5 & . & 15 & N2AB1 \\
	 & 6 & . & 16 & B5BB \\
	 & 7 & . & 17 & V10B \\
	 & 8 & . & 18 & N2AA1 \\
	 & 9 & . & 19 & N2AA1, N2AA1 \\
	 & 10 & . & 20 & N2A, Z9OP, Z9SA \\
	\addlinespace
	8 & 1 & C1BA & 7 & V4CB \\
	 & 2 & . & 8 & V4CB \\
	 & 3 & A7AA, D1AA, G1AA & 9 & . \\
	 & 4 & V4CB & 10 & A12AA12 \\
	 & 5 & . & 11 & R1AX \\
	 & 6 & J1CE2, J1CE1 & 12 & C2N \\
	\bottomrule
\end{tabular}
\end{table}

\begin{table}[htbp] \footnotesize \center
\caption{Task 2 B, chapter *T1.10, T2.2.5.1, T3.1, and T6.2.3\label{tab:task2b}}
\begin{tabular}{c c l c l}
    \toprule
    Chapter & Sentence & ATC codes & Sentence & ATC codes \\
    \midrule
	*T1.10 & 1 & C5A, D5A, A1AB & 6 & J1EE1, J1RA1 \\
	 & 2 & V7, V7A & 7 & C10AC, N5BA1, M4AB \\
	 & 3 & V9D, J1EB, A7EC1 & 8 & A5AB, D5BB, D10AF \\
	 & 4 & J7BC20, A7EC1, N4AC & 9 & V4CB, V4CD, V4CG \\
	 & 5 & B1AD12, B5A, C8D & 10 & J7BC20, A7EC1 \\
	\addlinespace
	T2.2.5.1 & 1 & M4AA, L3AB1, L3AB4 & 2 & A5AB, D5BB, D10AF \\
	\addlinespace
	T3.1 & 1 & C2LG51, A10X, V3AA & 23 & V3AH \\
	 & 2 & C10AC, M4AB, H2AB & 24 & H4AA1, Z0CA, A5AB \\
	 & 3 & B5D, B5BA3, B5CX1 & 25 & A5AB, D5BB, D10AF \\
	 & 4 & A10BX2, A10BX3, Z0ET & 26 & B5BA3, B5CX1, V4CA2 \\
	 & 5 & B5BB, G3, M5B & 27 & V9DX1 \\
	 & 6 & V6DB & 28 & A14A \\
	 & 7 & A10AD1 & 29 & B5BC2, D2AE1, B5BA3 \\
	 & 8 & A10BA2, Z9MF, A10BD2 & 30 & A10AE, B5BB3, B5BB \\
	 & 9 & A10AE & 31 & V7AD \\
	 & 10 & A10BD, J7BC20, A10AD1 & 32 & A10AB1, A10AD4 \\
	 & 11 & V3AH & 33 & . \\
	 & 12 & A7EC1, N4AC & 34 & V3AH \\
	 & 13 & B5BA3, B5CX1, V4CA2 & 35 & A10X, N4AC, N7 \\
	 & 14 & A10AE & 36 & C5BA, C5AX, C5BB \\
	 & 15 & A10BD3 & 37 & Z9AC \\
	 & 16 & A10AE4, A10AE5 & 38 & Z9ST, V9G \\
	 & 17 & A10AE & 39 & A5AX, M4AC, A7EC1 \\
	 & 18 & A10AC1, N4AC & 40 & Z9A2 \\
	 & 19 & C9AA5, A10AB, A10B & 41 & C10AC, M4AB, A10BA2 \\
	 & 20 & A10AC1, A10AB & 42 & . \\
	 & 21 & J4AK, S1KX, P1A & 43 & Z9AC \\
	 & 22 & V3AH, J4AK, M9AX & 44 & J4AK, V4CB, V4CD \\
	 & & & 45 & A1AB \\
	\addlinespace
	T6.2.3 & 1 & V3AA, N4AC & 4 & V9G, V4CB, V4CD \\
	 & 2 & S3, V4CJ, A7EC1 & 5 & V3AA, N3AX12, Z9BD \\
	 & 3 & N4AC & & \\
	\bottomrule
\end{tabular}
\end{table}


\section{Task 3: Ranking using vector models}
%--------------------------------------------
Ranked lists of relevant therapy chapters for each patient case can be found
in \autoref{tab:task3a} and \autoref{tab:task3b}.
\begin{table}[htbp] \footnotesize \center
\caption{Task 3 results (part 1)\label{tab:task3a}}
\begin{tabular}{c c l}
    \toprule
    Case & Rank & Relevant chapter \\
    \midrule
	1 & 1 & T3.1: Diabetes mellitus \\
	 & 2 & T10.2.1: Bronkial astma \\
	 & 3 & T14.5.1: Polycystisk ovarialt syndrom (PCOS) \\
	 & 4 & T23.1.1.2: Faste og stress \\
	 & 5 & T5.4.1: Schizofreni \\
	 & 6 & T14.2.1: Forskyvning av normal menstruasjon \\
	 & 7 & T10.2.1.1: Mild og moderat astma \\
	 & 8 & T18.1.4: Kontroll og oppfølging \\
	 & 9 & T16.13.1: Generalisert kløe \\
	 & 10 & *T9.1.5: Anafylaktoide reaksjoner \\
	\addlinespace
	2 & 1 & T10.2: Obstruktiv lungesykdom \\
	 & 2 & T10.2.2: Kronisk obstruktiv lungesykdom (kols) \\
	 & 3 & T8.4.1.2.2: Atrioventrikulær nodal reentrytakykardi (nodal takykardi) \\
	 & 4 & T10.2.1: Bronkial astma \\
	 & 5 & T8.3.2.2: Hjerteinfarkt med ST-elevasjon \\
	 & 6 & T3.1: Diabetes mellitus \\
	 & 7 & T10.8: Sarkoidose \\
	 & 8 & T15.3.7: Liten melkeproduksjon \\
	 & 9 & T5.3.1.3: Alkohol abstinensreaksjoner \\
	 & 10 & T6.2.2: Klasehodepine («Cluster headache», Hortons hodepine) \\
	\addlinespace
	3 & 1 & *T1.10: Akutt bakteriell meningitt \\
	 & 2 & T3.1: Diabetes mellitus \\
	 & 3 & T8.1: Hypertensjon \\
	 & 4 & T1.11: Bakteriell endokarditt \\
	 & 5 & T16.7.1: Skabb \\
	 & 6 & T8.2.1: Malign hypertensjon \\
	 & 7 & T19.1: Feber \\
	 & 8 & T8.2.2: Hypertensjonsencefalopati \\
	 & 9 & T8.3.2.2: Hjerteinfarkt med ST-elevasjon \\
	 & 10 & T14.6.4: Akutt bekkeninfeksjon \\
	\addlinespace
	4 & 1 & T8.3: Koronarsykdom \\
	 & 2 & T11.1.1.4.7: Emosjonell rhinitt \\
	 & 3 & T8.2.4: Hypertensjonskrise og hjerteinfarkt eller ustabil angina \\
	 & 4 & T4.6.3: Arteriell trombose \\
	 & 5 & T8.3.1: Stabil koronarsykdom (stabil angina pectoris) \\
	 & 6 & T8.4.1.2: Paroksystisk supraventrikulær takykardi \\
	 & 7 & T10.2.2: Kronisk obstruktiv lungesykdom (kols) \\
	 & 8 & *T8.3.2.1: Ustabil angina/hjerteinfarkt uten ST-elevasjon \\
	 & 9 & T24.2.1.7: Myokardscintigrafi \\
	 & 10 & T15.3.7: Liten melkeproduksjon \\
	\bottomrule
\end{tabular}
\end{table}

\begin{table}[htbp] \footnotesize \center
\caption{Task 3 results (part 2)\label{tab:task3b}}
\begin{tabularx}{\textwidth}{c c X}
    \toprule
    Case & Rank & Relevant chapter \\
    \midrule
	5 & 1 & T12.10.1: Hemoroider \\
	 & 2 & T12.9.3: Dyschezi (rektumobstipasjon) \\
	 & 3 & T4.1: Anemier \\
	 & 4 & T1.6.2.1: Clostridium difficile enterokolitt \\
	 & 5 & T12.10.3: Fissura ani \\
	 & 6 & T12.11: Familiær adenomatøs polypose \\
	 & 7 & T5.5: Depresjoner \\
	 & 8 & T15.1.5: Svangerskapsindusert hypertensjon \\
	 & 9 & T4.1.3.2: Talassemi \\
	 & 10 & T13.2.5: Nevrogene blæreforstyrrelser \\
	\addlinespace
	6 & 1 & T2.2.5.1: Cancer i nyreparenkym og binyre \\
	 & 2 & T11.3.2.2: Kronisk tonsillitt \\
	 & 3 & T11.3.1.2: Kronisk faryngitt \\
	 & 4 & T1.7.7: Lymfogranuloma venereum \\
	 & 5 & T11.4.4: Halitosis \\
	 & 6 & T1.1.8: Skarlagensfeber \\
	 & 7 & T11.3.2.1: Akutt tonsillitt \\
	 & 8 & T1.6.1: Ikke-inflammatoriske, toksinpregete enteritter \\
	 & 9 & T10.3.4: Pneumonier, bakterielle og med ukjent etiologi \\
	 & 10 & T10.2.1.1: Mild og moderat astma \\
	\addlinespace
	7 & 1 & T6.2.3: Spenningshodepine (Tensjonshodepine) \\
	 & 2 & T20.2.1: Akutte smerter \\
	 & 3 & T21.1.1.2: Nevropatiske smerter \\
	 & 4 & T20.2.3.1: Praktisk gjennomføring av smertebehandling hos pasienter med kort livsprognose \\
	 & 5 & T22.4.1.1: Postoperativ grunnanalgesi \\
	 & 6 & T20.1.2.2: Opioidanalgetika \\
	 & 7 & T20.2.2.1: Praktisk gjennomføring av smertebehandling hos pasienter med antatt normal levetid \\
	 & 8 & T21.1.1.1: Nociseptive smerter \\
	 & 9 & T20.2.3.2: Bruk av sterkere opioider hos pasienter med kort livsprognose \\
	 & 10 & T6.5.1: Multippel sklerose \\
	\addlinespace
	8 & 1 & T11.3.2.1: Akutt tonsillitt \\
	 & 2 & T1.1.8: Skarlagensfeber \\
	 & 3 & T1.7.5: Syfilis \\
	 & 4 & T11.3.1.1: Akutt faryngitt \\
	 & 5 & T1.3: Mononukleose \\
	 & 6 & *T1.10: Akutt bakteriell meningitt \\
	 & 7 & T16.5.1: Pyodermier \\
	 & 8 & T11.1.2.1: Akutt rhinosinusitt \\
	 & 9 & T10.3.4: Pneumonier, bakterielle og med ukjent etiologi \\
	 & 10 & T1.11: Bakteriell endokarditt \\
	\bottomrule
\end{tabularx}
\end{table}


\section{Task 4: Evaluation}
%---------------------------
An example of terms shared between retrieved therapy
chapter and patient case can be seen in \autoref{tab:terms}, where
relevant medical terms are boldfaced. The patient case concerns a patient with
diabetes mellitus and the first result ``T3.1 Diabetes mellitus'' is spot on,
while the second result ``T20.2.1. Bronkial astma'' is not relevant at all.
We have calculated average precision at ten documents seen (P@10) and
R-precision, which are listed in \autoref{tab:precision}.

Rank correlation metrics are an automatic way for comparing two ranking
methods, to determine how differently one varies from another. It does not
consider relevance of retrieved documents, only the relative ordering of two
rankings. \autoref{tab:kendalltau} list such a rank correlation metric, called
Kendall tau coefficients.

\begin{table}[htbp] \footnotesize \center
\caption{Task 4 shared terms (patient case 1)\label{tab:terms}}
\begin{tabularx}{\textwidth}{c l l c X}
    \toprule
    Rank & Chapter & Score & Relevant & Terms \\
    \midrule
	1 & T3.1 & 0.0832 & Yes & bruker, delvis, henvisning, hatt, \textbf{acetonlukt}, injeksjon, år, hurtigvirkende, hvert, \textbf{mellitus}, siste, lite, hurtig, \textbf{insulin}, håndtere, synes, flere, dessuten, vurderer, \textbf{diabetes}, kvelden, måltid, langtidsvirkende, \textbf{blodtrykk}, normalt, døgn, sykehus \\
	2 & T10.2.1 & 0.0429 & No & bruker, delvis, hurtig, flere, dessuten, vurderer \\
	3 & T14.5.1 & 0.0407 & Yes & \textbf{insulin}, uteblir \\
	4 & T23.1.1.2 & 0.0372 & Yes & lite, \textbf{insulin}, dessuten, normalt, døgn \\
	5 & T5.4.1 & 0.0372 & Yes & delvis, år, fått, hvert, lite, håndtere, synes, flere, \textbf{diabetes} \\
	6 & T14.2.1 & 0.0332 & No & bruker, siste, brukt, flere, tatt \\
	7 & T10.2.1.1 & 0.0325 & No & bruker, delvis, henvisning, hatt, år, fått, hvert, hurtig, synes, flere, langtidsvirkende, døgn \\
	8 & T18.1.4 & 0.0309 & No & hvert, kontroller \\
	9 & T16.13.1 & 0.0304 & Yes & tørr, huden, \textbf{mellitus}, flere, \textbf{diabetes} \\
	10 & *T9.1.5 & 0.0290 & Yes & \textbf{injiserer}, hatt, injeksjon, år, hurtig, \textbf{blodtrykk}, sykehus \\
	\bottomrule
\end{tabularx}
\end{table}

\begin{table}[htbp] \footnotesize \center
\caption{Task 4 precision of each patient case search\label{tab:precision}}
\begin{tabular}{c c c c c c c c c}
    \toprule
    & \multicolumn{4}{c}{Precision @ 10} & \multicolumn{4}{c}{R-precision} \\
	\cmidrule(r){2-9}
	Case & A & B & \textbf{C} & D & A & B & \textbf{C} & \textbf{D} \\
    \midrule
	1 & 60\% & 60\% & 70\% & 50\% & 0.67 & 0.67 & 0.71 & 0.60 \\
	2 & 50\% & 60\% & 70\% & 60\% & 0.80 & 0.67 & 0.71 & 0.83 \\
	3 & 90\% & 80\% & 80\% & 80\% & 0.89 & 0.88 & 0.75 & 0.75 \\
	4 & 70\% & 70\% & 60\% & 70\% & 0.71 & 0.86 & 0.83 & 0.86 \\
	5 & 80\% & 80\% & 90\% & 90\% & 0.88 & 0.88 & 0.89 & 0.89 \\
	6 & 80\% & 80\% & 80\% & 80\% & 0.75 & 0.75 & 0.88 & 0.88 \\
	7 & 100\% & 100\% & 100\% & 100\% & 1.00 & 1.00 & 1.00 & 1.00 \\
	8 & 90\% & 90\% & 80\% & 80\% & 0.89 & 0.89 & 0.88 & 0.88 \\
    \midrule
	Avg & 77.5\% & 77.5\% & \textbf{78.8\%} & 76.2\% & 0.82 & 0.82 & \textbf{0.83} & \textbf{0.83} \\
	\bottomrule
\end{tabular}
\end{table}

\begin{table}[htbp] \footnotesize \center
\caption{Task 4 Kendall tau coefficients\label{tab:kendalltau}}
\begin{tabular}{c c c c c c c}
    \toprule
	Case & A vs B & A vs C & A vs D & B vs C & B vs D & C vs D \\
    \midrule
	1 & 0.981 & 0.941 & 0.947 & 0.931 & 0.944 & 0.978 \\
	2 & 0.984 & 0.937 & 0.940 & 0.931 & 0.939 & 0.981 \\
	3 & 0.989 & 0.948 & 0.950 & 0.945 & 0.950 & 0.988 \\
	4 & 0.975 & 0.957 & 0.959 & 0.942 & 0.962 & 0.971 \\
	5 & 0.985 & 0.951 & 0.952 & 0.945 & 0.953 & 0.983 \\
	6 & 0.971 & 0.954 & 0.954 & 0.937 & 0.960 & 0.964 \\
	7 & 0.981 & 0.944 & 0.944 & 0.936 & 0.946 & 0.978 \\
	8 & 0.985 & 0.943 & 0.948 & 0.935 & 0.944 & 0.984 \\
    \midrule
	Avg & 0.981 & 0.947 & 0.949 & 0.938 & 0.950 & 0.979 \\
	\bottomrule
\end{tabular}
\end{table}


\section{Task 5: Exchange evaluations}
%-------------------------------------


\section{Task 6: Improving the ranking}
%--------------------------------------
Ranked lists of relevant therapy chapters for each patient case can be found
in \autoref{tab:task6a1} and \autoref{tab:task6a2}.
\begin{table}[htbp] \footnotesize \center
\caption{Task 6 A results (part 1)\label{tab:task6a1}}
\begin{tabular}{c c l}
    \toprule
    Case & Rank & Relevant chapter \\
    \midrule
	1 & 1 & T10.11: Respiratorisk distress-syndrom hos premature barn \\
	 & 2 & T3.1: Diabetes mellitus \\
	 & 3 & T12.4.2: Hemokromatose \\
	 & 4 & T12.10.2: Pruritus ani \\
	 & 5 & T12.2.2: Kronisk pankreatitt \\
	 & 6 & T16.13.1: Generalisert kløe \\
	 & 7 & T23.3.6.5: Korreksjon ved hyperton dehydrering \\
	 & 8 & T24.2.1.2: Dopamin transporter ligand scintigrafi \\
	 & 9 & T24.2.1.1: Cerebral blodstrøm \\
	 & 10 & T24.2.1.7: Myokardscintigrafi \\
	\addlinespace
	2 & 1 & T10.10: Cystisk fibrose \\
	 & 2 & T9.1: Allergiske lidelser \\
	 & 3 & T14.4: Endometriose \\
	 & 4 & T19.1: Feber \\
	 & 5 & T10.6: Tuberkulose \\
	 & 6 & T10.2: Obstruktiv lungesykdom \\
	 & 7 & T15.3.3: Brystabscess \\
	 & 8 & T15.3.1: Såre brystknopper \\
	 & 9 & T6.4.1: Essensiell tremor \\
	 & 10 & T12.5.4: Akutt kolangitt \\
	\addlinespace
	3 & 1 & T6.1.2: Feberkramper \\
	 & 2 & T1.2: Influensa \\
	 & 3 & T4.5.1: Hemofili A (faktor VIII mangel) og B (faktor IX mangel) \\
	 & 4 & T7.8.2: Glaukom med åpen kammervinkel \\
	 & 5 & T10.2: Obstruktiv lungesykdom \\
	 & 6 & T19.7: Leggkramper \\
	 & 7 & T10.4: Lungeabscess \\
	 & 8 & T20.2: Akutte og kroniske smerter \\
	 & 9 & T23.6: Instillasjon i urinblæren \\
	 & 10 & *T1.10: Akutt bakteriell meningitt \\
	\addlinespace
	4 & 1 & T19.1: Feber \\
	 & 2 & T14.4: Endometriose \\
	 & 3 & T1.1.7: Kikhoste \\
	 & 4 & T15.3.1: Såre brystknopper \\
	 & 5 & T15.3.7: Liten melkeproduksjon \\
	 & 6 & T8.3.1: Stabil koronarsykdom (stabil angina pectoris) \\
	 & 7 & T8.3.2.2: Hjerteinfarkt med ST-elevasjon \\
	 & 8 & T8.6: Hjertesvikt \\
	 & 9 & T9.1: Allergiske lidelser \\
	 & 10 & T5.5: Depresjoner \\
	\bottomrule
\end{tabular}
\end{table}

\begin{table}[htbp] \footnotesize \center
\caption{Task 6 A results (part 2)\label{tab:task6a2}}
\begin{tabular}{c c l}
    \toprule
    Case & Rank & Relevant chapter \\
    \midrule
	5 & 1 & T12.5.1: Gallestein \\
	 & 2 & T7.5.3: Spesielle uveitter \\
	 & 3 & T12.10.1: Hemoroider \\
	 & 4 & T12.2.1: Akutt pankreatitt \\
	 & 5 & T1.14.3: Piskemark, hakemark, strongyloidiasis \\
	 & 6 & T1.6.2.5: Campylobacteriose \\
	 & 7 & T1.6.2.3: Salmonelloser \\
	 & 8 & T1.6.2.1: Clostridium difficile enterokolitt \\
	 & 9 & T1.6.4.1: Giardiasis \\
	 & 10 & T1.6.4.3: Cyklosporidose \\
	\addlinespace
	6 & 1 & *T1.13: Nekrotiserende fasciitt \\
	 & 2 & T23.6: Instillasjon i urinblæren \\
	 & 3 & T3.1: Diabetes mellitus \\
	 & 4 & *T1.10: Akutt bakteriell meningitt \\
	 & 5 & T1.7.9: Herpes genitalis \\
	 & 6 & T10.3.4: Pneumonier, bakterielle og med ukjent etiologi \\
	 & 7 & T2.2.12: Myelomatose \\
	 & 8 & T1.7.6: Ulcus molle \\
	 & 9 & T4.1.1: Jernmangelanemi \\
	 & 10 & T11.4.10: Orale virusinfeksjoner \\
	\addlinespace
	7 & 1 & T19.4: Svimmelhet \\
	 & 2 & T6.1.2: Feberkramper \\
	 & 3 & T12.4.3: Porfyri \\
	 & 4 & T15.1.3: Bekkenløsning \\
	 & 5 & T14.1.2.1: Østrogen-gestagen til antikonsepsjon \\
	 & 6 & T24.2.2.3: Palliativ smertebehandling av skjelettmetastaser \\
	 & 7 & T5.4: Psykotiske lidelser (Psykoser) \\
	 & 8 & T8.12: Hyperlipidemi \\
	 & 9 & T16.9: Kutane bivirkninger av systemiske legemidler \\
	 & 10 & T14.4: Endometriose \\
	\addlinespace
	8 & 1 & T23.6: Instillasjon i urinblæren \\
	 & 2 & T12.5.4: Akutt kolangitt \\
	 & 3 & T1.7.6: Ulcus molle \\
	 & 4 & T15.3.3: Brystabscess \\
	 & 5 & T1.6.2.1: Clostridium difficile enterokolitt \\
	 & 6 & T1.12: Osteomyelitt \\
	 & 7 & T1.17: Desinfeksjon \\
	 & 8 & *T1.13: Nekrotiserende fasciitt \\
	 & 9 & *T1.10: Akutt bakteriell meningitt \\
	 & 10 & T10.10: Cystisk fibrose \\
	\bottomrule
\end{tabular}
\end{table}


\section{Task 7: Gold standard}
%------------------------------


