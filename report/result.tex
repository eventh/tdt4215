%---------------
\chapter{Result}
%---------------
This chapter presents our results from parsing input files and project tasks.


\section{Preprocessing and parsing}
%----------------------------------
The results of parsing the different input files can be seen in
\autoref{tab:objects}. Each code, chapter and case is stored in an object,
and saved to JSON files.
\begin{table}[htbp] \footnotesize \center
\caption{Parsed object counts\label{tab:objects}}
\begin{tabular}{l r}
    \toprule
    Type & Count \\
    \midrule
	ATC codes & 7945 \\
	ICD10 codes & 10521 \\
	Patient cases & 8 \\
	Therapy chapters & 917 \\
	\bottomrule
\end{tabular}
\end{table}

\autoref{tab:chapters} contains statistics showing the results of parsing
``Norsk legemiddelhåndbok'' HTML files --- therapy chapters. The statistics
lists the amount of chapters, in total and with text, for each of the
different chapter types --- from chapter to subsubsubsubchapter.
%Total amount of lines '\n': 3034
%Total amount of sentences '.': 24924
\begin{table}[htbp] \footnotesize \center
\caption{Norsk legemiddelhåndbok statistics\label{tab:chapters}}
\begin{tabular}{l r r}
    \toprule
    Chapter type & Count & With text \\
    \midrule
	Chapter & 24 & 24 \\
	Subchapter & 153 & 104 \\
	Subsubchapter & 384 & 336 \\
	Subsubsubchapter & 329 & 326 \\
	Subsubsubsubchapter & 27 & 27 \\
    \midrule
	Total & 917 & 817 \\
	\bottomrule
\end{tabular}
\end{table}

Parsing of patient cases are summarized in \autoref{tab:cases}. Stopwords
refers to stopwords in the case text which have been removed, terms is the
number of unqiue terms (words) in the text.
\begin{table}[htbp] \footnotesize \center
\caption{Patient cases statistics\label{tab:cases}}
\begin{tabular}{c r r r r}
    \toprule
	Case \# & Lines & Stopwords & Terms & Medical terms \\
    \midrule
	1 & 13 & 1 & 55 & 10 \\
	2 & 22 & 1 & 169 & 24 \\
	3 & 17 & 0 & 125 & 30 \\
	4 & 6 & 1 & 49 & 4 \\
	5 & 11 & 2 & 63 & 14 \\
	6 & 7 & 0 & 41 & 9 \\
	7 & 20 & 2 & 123 & 29 \\
	8 & 12 & 0 & 90 & 19 \\
    \midrule
	Total & 108 & 7 & 715 & 125 \\
	\bottomrule
\end{tabular}
\end{table}

We store input data and results in JSON format, so we can work with them
without having to parse or produce them first. To demonstrate the
effectiveness of this method we list and compare the time (in seconds) it
takes to parse input data and loading JSON files in \autoref{tab:times}.
\begin{table}[htbp] \footnotesize \center
\caption{Comparing effectivenes of JSON\label{tab:times}}
\begin{tabular}{l r r c}
    \toprule
    Type & Parse time & JSON load time & Speedup \\
    \midrule
	ATC codes & 0.209 & 0.101 & 52\% \\
	ICD10 codes & 9.496 & 0.549 & 94\% \\
	Patient cases & 0.008 & 0.001 & 88\% \\
	Therapy chapters & 11.326 & 0.177 & 98\% \\
	\bottomrule
\end{tabular}
\end{table}


\section{Task 1 A}
%-----------------
Describe with tables etc short results, for example one single case.
Also show ranking and matched terms and score for same case.


\section{Task 1 B}
%-----------------


\section{Task 2 A}
%-----------------


\section{Task 2 B}
%-----------------


\section{Task 3}
%---------------


\section{Task 4}
%---------------


\section{Task 5}
%---------------


\section{Task 6 A}
%-----------------


\section{Task 6 B}
%-----------------


\section{Task 7}
%-----------------


