\begin{table}[htbp] \footnotesize \center
\caption{Task 3\label{tab:task3a}}
\begin{tabularx}{\textwidth}{c c X}
    \toprule
    Clinical note & Result & Relevant chapter \\
    \midrule
	1 & 1 & T3.1: Diabetes mellitus \\
	 & 2 & T10.2.1: Bronkial astma \\
	 & 3 & T14.5.1: Polycystisk ovarialt syndrom (PCOS) \\
	 & 4 & T23.1.1.2: Faste og stress \\
	 & 5 & T5.4.1: Schizofreni \\
	 & 6 & T14.2.1: Forskyvning av normal menstruasjon \\
	 & 7 & T10.2.1.1: Mild og moderat astma \\
	 & 8 & T18.1.4: Kontroll og oppfølging \\
	 & 9 & T16.13.1: Generalisert kløe \\
	 & 10 & *T9.1.5: Anafylaktoide reaksjoner \\
	\addlinespace
	2 & 1 & T10.2: Obstruktiv lungesykdom \\
	 & 2 & T10.2.2: Kronisk obstruktiv lungesykdom (kols) \\
	 & 3 & T8.4.1.2.2: Atrioventrikulær nodal reentrytakykardi (nodal takykardi) \\
	 & 4 & T10.2.1: Bronkial astma \\
	 & 5 & T8.3.2.2: Hjerteinfarkt med ST-elevasjon \\
	 & 6 & T3.1: Diabetes mellitus \\
	 & 7 & T10.8: Sarkoidose \\
	 & 8 & T15.3.7: Liten melkeproduksjon \\
	 & 9 & T5.3.1.3: Alkohol abstinensreaksjoner \\
	 & 10 & T6.2.2: Klasehodepine («Cluster headache», Hortons hodepine) \\
	\addlinespace
	3 & 1 & *T1.10: Akutt bakteriell meningitt \\
	 & 2 & T3.1: Diabetes mellitus \\
	 & 3 & T8.1: Hypertensjon \\
	 & 4 & T1.11: Bakteriell endokarditt \\
	 & 5 & T16.7.1: Skabb \\
	 & 6 & T8.2.1: Malign hypertensjon \\
	 & 7 & T19.1: Feber \\
	 & 8 & T8.2.2: Hypertensjonsencefalopati \\
	 & 9 & T8.3.2.2: Hjerteinfarkt med ST-elevasjon \\
	 & 10 & T14.6.4: Akutt bekkeninfeksjon \\
	\addlinespace
	4 & 1 & T8.3: Koronarsykdom \\
	 & 2 & T11.1.1.4.7: Emosjonell rhinitt \\
	 & 3 & T8.2.4: Hypertensjonskrise og hjerteinfarkt eller ustabil angina \\
	 & 4 & T4.6.3: Arteriell trombose \\
	 & 5 & T8.3.1: Stabil koronarsykdom (stabil angina pectoris) \\
	 & 6 & T8.4.1.2: Paroksystisk supraventrikulær takykardi \\
	 & 7 & T10.2.2: Kronisk obstruktiv lungesykdom (kols) \\
	 & 8 & *T8.3.2.1: Ustabil angina/hjerteinfarkt uten ST-elevasjon \\
	 & 9 & T24.2.1.7: Myokardscintigrafi \\
	 & 10 & T15.3.7: Liten melkeproduksjon \\
	\bottomrule
\end{tabularx}
\end{table}

\begin{table}[htbp] \footnotesize \center
\caption{Task 3\label{tab:task3b}}
\begin{tabularx}{\textwidth}{c c X}
    \toprule
    Clinical note & Result & Relevant chapter \\
    \midrule
	5 & 1 & T12.10.1: Hemoroider \\
	 & 2 & T12.9.3: Dyschezi (rektumobstipasjon) \\
	 & 3 & T4.1: Anemier \\
	 & 4 & T1.6.2.1: Clostridium difficile enterokolitt \\
	 & 5 & T12.10.3: Fissura ani \\
	 & 6 & T12.11: Familiær adenomatøs polypose \\
	 & 7 & T5.5: Depresjoner \\
	 & 8 & T15.1.5: Svangerskapsindusert hypertensjon \\
	 & 9 & T4.1.3.2: Talassemi \\
	 & 10 & T13.2.5: Nevrogene blæreforstyrrelser \\
	\addlinespace
	6 & 1 & T2.2.5.1: Cancer i nyreparenkym og binyre \\
	 & 2 & T11.3.2.2: Kronisk tonsillitt \\
	 & 3 & T11.3.1.2: Kronisk faryngitt \\
	 & 4 & T1.7.7: Lymfogranuloma venereum \\
	 & 5 & T11.4.4: Halitosis \\
	 & 6 & T1.1.8: Skarlagensfeber \\
	 & 7 & T11.3.2.1: Akutt tonsillitt \\
	 & 8 & T1.6.1: Ikke-inflammatoriske, toksinpregete enteritter \\
	 & 9 & T10.3.4: Pneumonier, bakterielle og med ukjent etiologi \\
	 & 10 & T10.2.1.1: Mild og moderat astma \\
	\addlinespace
	7 & 1 & T6.2.3: Spenningshodepine (Tensjonshodepine) \\
	 & 2 & T20.2.1: Akutte smerter \\
	 & 3 & T21.1.1.2: Nevropatiske smerter \\
	 & 4 & T20.2.3.1: Praktisk gjennomføring av smertebehandling hos pasienter med kort livsprognose \\
	 & 5 & T22.4.1.1: Postoperativ grunnanalgesi \\
	 & 6 & T20.1.2.2: Opioidanalgetika \\
	 & 7 & T20.2.2.1: Praktisk gjennomføring av smertebehandling hos pasienter med antatt normal levetid \\
	 & 8 & T21.1.1.1: Nociseptive smerter \\
	 & 9 & T20.2.3.2: Bruk av sterkere opioider hos pasienter med kort livsprognose \\
	 & 10 & T6.5.1: Multippel sklerose \\
	\addlinespace
	8 & 1 & T11.3.2.1: Akutt tonsillitt \\
	 & 2 & T1.1.8: Skarlagensfeber \\
	 & 3 & T1.7.5: Syfilis \\
	 & 4 & T11.3.1.1: Akutt faryngitt \\
	 & 5 & T1.3: Mononukleose \\
	 & 6 & *T1.10: Akutt bakteriell meningitt \\
	 & 7 & T16.5.1: Pyodermier \\
	 & 8 & T11.1.2.1: Akutt rhinosinusitt \\
	 & 9 & T10.3.4: Pneumonier, bakterielle og med ukjent etiologi \\
	 & 10 & T1.11: Bakteriell endokarditt \\
	\bottomrule
\end{tabularx}
\end{table}


